\documentclass[a4paper]{scrartcl}

% font/encoding packages
\usepackage[utf8]{inputenc}
\usepackage[T1]{fontenc}
\usepackage{lmodern}
\usepackage[ngerman]{babel}
\usepackage[ngerman=ngerman-x-latest]{hyphsubst}

\usepackage{amsmath, amssymb, amsfonts, amsthm}
\usepackage{array}
\usepackage{stmaryrd}
\usepackage{marvosym}
\allowdisplaybreaks
\usepackage[output-decimal-marker={,}]{siunitx}
\usepackage[shortlabels]{enumitem}
\usepackage[section]{placeins}
\usepackage{float}
\usepackage{units}
\usepackage{listings}
\usepackage{pgfplots}
\pgfplotsset{compat=1.12}
\usepackage{tikz}
\usetikzlibrary{arrows,automata}


\newtheorem*{behaupt}{Behauptung}
\newcommand{\gdw}{\Leftrightarrow}
\newcommand{\dif}{\ \mathrm{d}}
\newcommand{\N}{\mathbb{N}}
\newcommand{\prob}{\mathbb{P}}
\newcommand{\cov}{\operatorname{Cov}}
\newcommand{\e}{\mathbb{E}}
\newcommand{\var}{\operatorname{Var}}
\newcommand{\corr}{\operatorname{Corr}}

\usepackage{fancyhdr}
\pagestyle{fancy}

\lstset{%
    frame=single,
    numbers=left,
    keepspaces,
    language=R,
    title=Listing: \lstname,
}

\def \blattnr {9}

\lhead{Stochastik 2 - Blatt {\blattnr}}
\rhead{Florian Abt, Lennart Braun, Sascha Schulz}
\cfoot{\thepage}


\title{Stochastik 2 für Informatiker}
\subtitle{Blatt {\blattnr} Hausaufgaben}
\author{
    Florian Abt (6524404), \\
    Lennart Braun (6523742), \\
    Sascha Schulz (6434677)
}
\date{zum 15. Dezember 2015}

\begin{document}
\maketitle

\begin{enumerate}[label=\bfseries \blattnr.\arabic*]
    \item
        \begin{enumerate}
            \item
                \begin{equation*}
                    H_0: \mathbb{P} \in \mathcal{P}_0 \text{ mit }
                    \mathcal{P}_0 =
                    \{ \mathcal{N}(\mu,\sigma^2) \ |\ 
                    \mu,\sigma^2 \in \mathbb{R} \land \sigma^2 > 0 \}
                \end{equation*}

            \item 
                \begin{equation*}
                    H_0: \mathbb{P} \in \mathcal{P}_0 \text{ mit }
                    \mathcal{P}_0 =
                    \{ \mathcal{N}(0,1) \}
                \end{equation*}

            \item 
                % Da $\mathcal{P}$ als Menge aller
                % Wahrscheinlichkeitsverteilungen mit \textbf{stetiger
                % Verteilungsfunktion} angegeben wird, existiert zu einer jeder
                % dieser Verteilungsfunktionen auch eine Dichte.
                % \begin{equation*}
                %     H_0: \mathbb{P} \in \mathcal{P}_0; \mathcal{P}_0 = \left\{
                %     \mathbb{P} : F(x)=\mathbb{P}(T_n(X_1,\ldots,X_n) \leq x)
                %     \wedge 3 \leq \int_{-\infty}^{\infty} F'(x) \dif x \right\}
                % \end{equation*}
                % Das Integral müsste = 1 sein.

                \begin{equation*}
                    H_0: \mathbb{P} \in \mathcal{P}_0 \text{ mit }
                    \mathcal{P}_0 =
                    \{ \prob_0 \in \mathcal{P}\ |\ 
                    \forall X \sim \prob_0 : \e[X] \geq 3 \}
                \end{equation*}

           \item 
                \begin{equation*}
                    H_0: \mathbb{P} \in \mathcal{P}_0 \text{ mit }
                    \mathcal{P}_0 =
                    \{ \prob_0 \in \mathcal{P}\ |\ 
                    \forall X \sim \prob_0 : \var[X] = 5 \}
                \end{equation*}
      \end{enumerate}

    \item 
        Wähle die Hypothesen so, dass ein Fehler 1. Art die ungünstigere
        Fehlentscheidung ist, deren Wahrscheinlichkeit möglichst gering sein
        soll.
        \begin{equation*}
            H_0 \colon \mu \geq \mu_0
            \quad \text{und} \quad
            H_1 \colon \mu < \mu_0
        \end{equation*}
        Konsequenz eines Fehlers 1. Art (die Nullhypothese wird zu unrecht
        abgelehnt, $H_0$ gilt):
        Die ehemaligen Bewohner werden in die verstrahlten Gebiete gelassen,
        obwohl die mittlere Strahlung über dem Grenzwert liegt. Die Gesundheit
        der Menschen kann beeinträchtigt werden und die Regierung müsste
        eventuell Entschädigungen zahlen. (Dies möchte man vermeiden.)

        Konsequenz eines Fehlers 2. Art (die Nullhypothese wird nicht abgelehnt,
        obwohl $H_1$ gilt):
        Die mittlere Strahlung wird überschätzt und die Menschen dürfen keine
        Besitztümer aus dem verstrahlten Gebiet retten und haben Pech gehabt.
        
        \begin{tabular}{c|c|c}
            & Entscheide „Gefahr“ & Entscheide „keine Gefahr“ \\
            \hline
            Ist „Gefahr“           & -             & Fehler 1. Art \\
            Ist „keine Gefahr“     & Fehler 2. Art & - 
        \end{tabular}
        
        Auf Grund des ethischen Quasi-Konsens unserer Gesellschaft wird die
        Konsequenz eines Fehlers erster Art als deutlich verwerflicher als die
        Konsequenz eines Fehlers zweiter Art bewertet, weshalb Fehler der
        ersten Art zu minimieren sind ($\Rightarrow$ entsprechende Wahl der
        Nullhypothese).
    
    \item 
        \begin{enumerate}
            \item
                Für $H_0\colon \mu = \mu_0$ ist der kritische Bereich
                \begin{equation*}
                    K = \left\{
                            x \in \mathbb{R}\ :\ 
                            |x| > z_{1-\frac{\alpha}{2}}
                    \right\}
                    = \left( -\infty, z_{\frac{\alpha}{2}} \right) \cup
                    \left( z_{1-\frac{\alpha}{2}}, \infty \right)
                \end{equation*}
                Somit gilt
                \begin{gather*}
                    H_0 \text{ wird abgelehnt} \\
                    \gdw\ T_n \in K \\
                    \gdw\ T_n < z_{\frac{\alpha}{2}} \lor T_n > z_{1-\frac{\alpha}{2}} \\
                    \gdw\ \Phi(T_n) < \frac{\alpha}{2} \lor \Phi(T_n) > 1-\frac{\alpha}{2} \\
                    \gdw\ 2 \cdot \Phi(T_n) < \alpha \lor 2(1-\Phi(T_n)) < \alpha \\
                    \gdw\ \min \Big(2 \cdot \Phi(T_n),\ 2(1-\Phi(T_n)) \Big) < \alpha
                \end{gather*}
                Daraus ergib sich als $p$-Wert:
                \begin{equation*}
	                \begin{split}
	                    p 
                        &= \min \Big(2 \cdot \Phi(T_n),\ 2(1-\Phi(T_n)) \Big) \\
                        &= 2 \cdot \min \Big(\Phi(T_n),\ 1-\Phi(T_n) \Big)
	                \end{split}
	            \end{equation*}
       
            \item 
                Es treten 700 von 820 Ergebnissen in $\{0, \dotsc, 30\}$ und
                120 von 820 Ergebnissen in $\{31, \dotsc, 36\}$ auf.  Wäre das
                Roulette fair, so würden die Einzelergebnisse Laplace-verteilt
                sein, d.\,h. $\prob(X_1 = x) = \frac{1}{37}$ für $x \in \{0,
                \dotsc, 36\}$.
                Wir betrachten nun ein Ergebnis in $\{31, \dotsc, 36\}$ als
                Erfolg. Wäre das Roulette fair, so wäre die
                Erfolgswahrscheinlichkeit $p_0 = \frac{6}{37}$.

                Wir testen nun die Hypothese $H_0 \colon p = p_0 =
                \frac{6}{37}$ gegen $H_1 \colon p \neq \frac{6}{37}$ mit einem
                Signifikanzniveau von $\alpha = \num{0,05}$.
                Der kritische Bereich ist
                \begin{equation*}
                    \begin{split}
                        K
                        &= \{ x \in \mathbb{R}\ :\ |x| > z_{1-\frac{\alpha}{2}} \} \\
                        &= \{ x \in \mathbb{R}\ :\ |x| > z_{\num{0,975}} \} \\
                        &= \Big( -\infty, \num{-1,960} \Big) \cup \Big( \num{1,960}, \infty \Big)
                    \end{split}
                \end{equation*}
                Der konkrete Wert des Schätzers $T_n$ für die gegebene
                Stichprobe beträgt
                \begin{equation*}
                    T_n
                    = \frac{120 - 820 \cdot \frac{6}{37}}{\sqrt{820 \cdot \frac{6}{37} \cdot \frac{31}{37}}}
                    = \num{-1,2291}
                    \text{ .}
                \end{equation*}
                Da der Schätzwert nicht im kritischen Bereich $K$ liegt, wird
                die Nullhypothese nicht abgelehnt. Es ist nicht genug, um das
                Roulette als unfair einzustufen.
	
        \end{enumerate}

    \pagebreak
    \item \hfill \\ 
      \lstinputlisting{aufgabe-9.4.r}
      
      \pagebreak
      Ergebnisse:
      
      \begin{tabular}{c|r|r}
       $H_0$ & rel. Häuf. Fehler 1. Art & rel. Häuf. Fehler 2. Art \\
       \hline
       $\mu \geq 49$ & 0.0322 & 0 \\
       $\mu \geq 51$ & 0 & 0.7738
      \end{tabular}
      
      Da wir wissen, dass tatsächlich ein $\mu=50$ vorliegt, ist
      offensichtlich, dass ausschließlich Fehler einer Art auftreten, eben
      abhängig davon, ob $H_0$ oder $H_1$ Übereinstimmung mit der Wirklichkeit
      hat.
      
      In beiden Szenarien wird für die Nullhypothese ein Wert für $\mu$
      gewählt, der um 1 vom tatsächlichen Wert abweicht. Am Ergebnis lässt sich
      sehr gut erkennen, mit was für einer stark unterschiedlichen Häufigkeit
      die Fehler im jeweiligen Szenario auftreten. Wird $\mu$ durch die
      Nullhypothese um 1 unterschätzt, erhalten wir lediglich in
      $\SI{3.22}{\percent} < \alpha$ der Fälle einen Fehler der ersten Art, da
      dieser Fall optimiert ist. Im Gegenzug erhalten wir in
      \SI{77.38}{\percent} der Fälle einen Fehler der zweiten Art, wenn wir
      $\mu$ um 1 überschätzen, was einen Fehler in deutlich mehr als
      \SI{50}{\percent} der Fälle ist, also überhaupt keinen Rückschluss auf
      die Wahrhaftigkeit von $\mu$ zulässt.

\end{enumerate}

\end{document}

