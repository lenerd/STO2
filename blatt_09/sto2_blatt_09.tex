\documentclass[a4paper]{scrartcl}

% font/encoding packages
\usepackage[utf8]{inputenc}
\usepackage[T1]{fontenc}
\usepackage{lmodern}
\usepackage[ngerman]{babel}
\usepackage[ngerman=ngerman-x-latest]{hyphsubst}

\usepackage{amsmath, amssymb, amsfonts, amsthm}
\usepackage{array}
\usepackage{stmaryrd}
\usepackage{marvosym}
\allowdisplaybreaks
\usepackage[output-decimal-marker={,}]{siunitx}
\usepackage[shortlabels]{enumitem}
\usepackage[section]{placeins}
\usepackage{float}
\usepackage{units}
\usepackage{listings}
\usepackage{pgfplots}
\pgfplotsset{compat=1.12}
\usepackage{tikz}
\usetikzlibrary{arrows,automata}


\newtheorem*{behaupt}{Behauptung}
\newcommand{\gdw}{\Leftrightarrow}
\newcommand{\dif}{\ \mathrm{d}}
\newcommand{\N}{\mathbb{N}}
\newcommand{\prob}{\mathbb{P}}
\newcommand{\cov}{\operatorname{Cov}}
\newcommand{\e}{\mathbb{E}}
\newcommand{\var}{\operatorname{Var}}
\newcommand{\corr}{\operatorname{Corr}}

\usepackage{fancyhdr}
\pagestyle{fancy}

\lstset{%
    frame=single,
    numbers=left,
    keepspaces,
    language=R,
    title=Listing: \lstname,
}

\def \blattnr {9}

\lhead{Stochastik 2 - Blatt {\blattnr}}
\rhead{Florian Abt, Lennart Braun, Sascha Schulz}
\cfoot{\thepage}


\title{Stochastik 2 für Informatiker}
\subtitle{Blatt {\blattnr} Hausaufgaben}
\author{
    Florian Abt (6524404), \\
    Lennart Braun (6523742), \\
    Sascha Schulz (6434677)
}
\date{zum 15. Dezember 2015}

\begin{document}
\maketitle

\begin{enumerate}[label=\bfseries \blattnr.\arabic*]

    \item 
      \begin{enumerate}
       \item 
        \begin{equation*}
	  H_0: \mathbb{P} \in \mathcal{P}_0; \mathcal{P}_0 = \{ \mathcal{N}(\mu,\sigma^2) \}
	\end{equation*}
	Mit beliebigen Werten für $\mu$ und $\sigma$.
       \item 
	\begin{equation*}
	  H_0: \mathbb{P} \in \mathcal{P}_0; \mathcal{P}_0 = \{ \mathcal{N}(0,1) \}
	\end{equation*}
       \item 
	Da $\mathcal{P}$ als Menge aller Wahrscheinlichkeitsverteilungen mit 
	\textbf{stetiger Verteilungsfunktion} angegeben wird, existiert zu einer 
	jeder dieser Verteilungsfunktionen auch eine Dichte.
	\begin{equation*}
	  H_0: \mathbb{P} \in \mathcal{P}_0; \mathcal{P}_0 = 
	   \left\{ \mathbb{P} : F(x)=\mathbb{P}(T_n(X_1,\ldots,X_n) \leq x) \wedge 3 \leq \int_{-\infty}^{\infty} F'(x) \dif x \right\}
	\end{equation*}
       \item 
      \end{enumerate}

    \item 
      Wähle die Nullhypothese als Gefahr, die mit möglichst nicht verkannt werden soll.
      
      $H_0: \mu \geq \mu_0$, und somit $H_1: \mu>\mu_0$.
      
      \begin{tabular}{c|c|c}
        & Entscheide ``Gefahr'' & Entscheide ``keine Gefahr'' \\
        \hline
       Ist ``Gefahr'' & - & Fehler 1. Art \\
       Ist ``keine Gefahr'' & Fehler 2. Art & - 
      \end{tabular}

    
      Konsequenz eines Fehlers 1. Art: Der tatsächliche mittlere Strahlenwert wurde unterschätzt und das Betreten wurde zu unrecht 
      gestattet. Dies kann enorme Gesundheits-Schäden durch die Strahlenkrankheit, bis zu Todesopfern zur Folge haben.
      
      Konsequenz eines Fehlers 2. Art: Der tatsächliche mittlere Strahlenwert wurde übreschätzt und das Betreten wurde zu unrecht 
      verboten. Betroffene verlieren Besitztümer, die sie nicht retten dürfen. 
      
      Auf Grund des ethischen Quasi-Konsens unserer Gesellschaft wird die Konsequenz eines Fehlers erster Art als deutlich verwerflicher 
      als die Konsequenz eines Fehlers zweiter Art bewertet, weshalb Fehler der ersten Art zu minimieren sind ($\Rightarrow$ entsprechende 
      Wahl der Nullhypothese).
    
    \item 
      \begin{enumerate}
       \item Für $H_0: \mu = \mu_0$ ist 
       $K = \left\{x:|x>z_{1-\frac\alpha 2} \right\} = \left(-\infty, z_{\frac\alpha 2}\right) \cup \left(z_{1-\frac\alpha 2}, \infty \right)$. Und somit
	\begin{equation*}
	  \begin{split}
	    T_n<z_{\frac\alpha 2} \Leftrightarrow \Phi(T_n)<\frac\alpha 2 &\Rightarrow \alpha > 2\cdot \Phi(T_n) \\
	    t_n>t_{1-\frac\alpha 2} \Leftrightarrow \Phi(T_n) > 1-\frac\alpha 2 &\Rightarrow \alpha > 2(1-\Phi(T_n))
	  \end{split}
	\end{equation*}
      Daraus ergib sich als Schranke: 
      \begin{equation*}
	  \begin{split}
	    p 
	    &= \min\{2\cdot \Phi(T_n), 2(1-\Phi(T_n))\}  \\
	    &= 2 \cdot \min\{\Phi(T_n), 1-\Phi(T_n)\}
	  \end{split}
	\end{equation*}
       
       \item 
        700 von 820 Ergebnissen in [0,30] und \\
	120 von 820 Ergebnissen in [31,36].
	
	Fairness würde La-Place-Verteilung vorraussetzen, sodass gilt \\
	$$\forall x \in \{0,1,\ldots,36\}: \mathbb{P}(X_1 = x) = \frac1{37}$$
	\begin{equation*}
	  \begin{split}
	    P(k|n,p) = \binom{n}{k} p^ k(1-p)^{n-k} \\
	    P\left(700|820,\frac{31}{37}\right) \approx 0.0181 \\ 
	    P\left(120|820,\frac{6}{37}\right) \approx 0.0181
	  \end{split}
	\end{equation*}
	
      \end{enumerate}

    \pagebreak
    \item \hfill \\ 
      \lstinputlisting{aufgabe-9.4.r}
      
      \pagebreak
      Ergebnisse:
      
      \begin{tabular}{c|r|r}
       $H_0$ & rel. Häuf. Fehler 1. Art & rel. Häuf. Fehler 2. Art \\
       \hline
       $\mu \geq 49$ & 0.0322 & 0 \\
       $\mu \geq 51$ & 0 & 0.7738
      \end{tabular}
      
      Da wir wissen, dass tatsächlich ein $\mu=50$ vorliegt, ist offensichtlich, 
      dass ausschließlich Fehler einer Art auftreten, eben abhängig davon, ob $H_0$ 
      oder $H_1$ Übereinstimmung mit der Wirklichkeit hat.
      
      In beiden Szenarien wird für die Nullhypothese ein Wert für $\mu$ gewählt, der 
      um 1 vom tatsächlichen Wert abweicht. Am Ergebnis lässt sich sehr gut erkennen,
      mit was für einer stark unterschiedlichen Häufigkeit die Fehler im jeweiligen 
      Szenario auftreten. Wird $\mu$ durch die Nullhypothese um 1 unterschätzt, 
      erhalten wir lediglich in 3.22\% der Fälle einen Fehler der ersten Art, 
      da dieser Fall optimiert ist. Im Gegenzug erhalten wir in 77.38\% der Fälle 
      einen Fehler der zweiten Art, wenn wir $\mu$ um 1 überschätzen, was einen 
      Fehler in deutlich mehr als 50\% der Fälle ist, also überhaupt keinen Rückschluss auf 
      die Wahrhaftigkeit von $\mu$ zulässt.

\end{enumerate}

\end{document}

