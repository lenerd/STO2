\documentclass[a4paper]{scrartcl}

% font/encoding packages
\usepackage[utf8]{inputenc}
\usepackage[T1]{fontenc}
\usepackage{lmodern}
\usepackage[ngerman]{babel}
\usepackage[ngerman=ngerman-x-latest]{hyphsubst}

\usepackage{amsmath, amssymb, amsfonts, amsthm}
\usepackage{array}
\usepackage{stmaryrd}
\usepackage{marvosym}
\allowdisplaybreaks
\usepackage[output-decimal-marker={,}]{siunitx}
\usepackage[shortlabels]{enumitem}
\usepackage[section]{placeins}
\usepackage{float}
\usepackage{units}
\usepackage{listings}
\usepackage{pgfplots}
\pgfplotsset{compat=1.12}
\usepackage{tikz}
\usetikzlibrary{arrows,automata}


\newtheorem*{behaupt}{Behauptung}
\newcommand{\gdw}{\Leftrightarrow}
\newcommand{\dif}{\ \mathrm{d}}
\newcommand{\N}{\mathbb{N}}
\newcommand{\prob}{\mathbb{P}}
\newcommand{\cov}{\operatorname{Cov}}
\newcommand{\e}{\mathbb{E}}
\newcommand{\var}{\operatorname{Var}}
\newcommand{\corr}{\operatorname{Corr}}

\usepackage{fancyhdr}
\pagestyle{fancy}

\lstset{%
    frame=single,
    numbers=left,
    keepspaces,
    language=R,
    title=Listing: \lstname,
}

\def \blattnr {9}

\lhead{Stochastik 2 - Blatt {\blattnr}}
\rhead{Florian Abt, Lennart Braun, Sascha Schulz}
\cfoot{\thepage}


\title{Stochastik 2 für Informatiker}
\subtitle{Blatt {\blattnr} Hausaufgaben}
\author{
    Florian Abt (6524404), \\
    Lennart Braun (6523742), \\
    Sascha Schulz (6434677)
}
\date{zum 15. Dezember 2015}

\begin{document}
\maketitle

\begin{enumerate}[label=\bfseries \blattnr.\arabic*]

    \item 
      \begin{enumerate}
       \item 
        \begin{equation*}
	  H_0: \mathbb{P} \in \mathcal{P}_0; \mathcal{P}_0 = \{ \mathcal{N}(\mu,\sigma^2) \}
	\end{equation*}
	Mit beliebigen Werten für $\mu$ und $\sigma$.
       \item 
	\begin{equation*}
	  H_0: \mathbb{P} \in \mathcal{P}_0; \mathcal{P}_0 = \{ \mathcal{N}(0,1) \}
	\end{equation*}
       \item 
	Da $\mathcal{P}$ als Menge aller Wahrscheinlichkeitsverteilungen mit 
	\textbf{stetiger Verteilungsfunktion} angegeben wird, existiert zu einer 
	jeder dieser Verteilungsfunktionen auch eine Dichte.
	\begin{equation*}
	  H_0: \mathbb{P} \in \mathcal{P}_0; \mathcal{P}_0 = 
	   \left\{ \mathbb{P} : F(x)=\mathbb{P}(T_n(X_1,\ldots,X_n) \leq x) \wedge 3 \leq \int_{-\infty}^{\infty} F'(x) \dif x \right\}
	\end{equation*}
       \item 
      \end{enumerate}

    \item 
    
    \item 
      \begin{enumerate}
       \item 
       \item 
      \end{enumerate}

    \item 
      

\end{enumerate}

\end{document}

