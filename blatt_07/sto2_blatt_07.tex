\documentclass[a4paper]{scrartcl}

% font/encoding packages
\usepackage[utf8]{inputenc}
\usepackage[T1]{fontenc}
\usepackage{lmodern}
\usepackage[ngerman]{babel}
\usepackage[ngerman=ngerman-x-latest]{hyphsubst}

\usepackage{amsmath, amssymb, amsfonts, amsthm}
\usepackage{array}
\usepackage{stmaryrd}
\usepackage{marvosym}
\allowdisplaybreaks
\usepackage[output-decimal-marker={,}]{siunitx}
\usepackage[shortlabels]{enumitem}
\usepackage[section]{placeins}
\usepackage{float}
\usepackage{units}
\usepackage{listings}
\usepackage{pgfplots}
\pgfplotsset{compat=1.12}
\usepackage{tikz}
\usetikzlibrary{arrows,automata}


\newtheorem*{behaupt}{Behauptung}
\newcommand{\gdw}{\Leftrightarrow}
\newcommand{\dif}{\ \mathrm{d}}
\newcommand{\N}{\mathbb{N}}
\newcommand{\prob}{\mathbb{P}}
\newcommand{\cov}{\operatorname{Cov}}
\newcommand{\e}{\mathbb{E}}
\newcommand{\var}{\operatorname{Var}}
\newcommand{\corr}{\operatorname{Corr}}

\usepackage{fancyhdr}
\pagestyle{fancy}

\lstset{%
    frame=single,
    numbers=left,
    keepspaces,
    language=R,
    title=Listing: \lstname,
}

\def \blattnr {7}

\lhead{Stochastik 2 - Blatt {\blattnr}}
\rhead{Florian Abt, Lennart Braun, Sascha Schulz}
\cfoot{\thepage}


\title{Stochastik 2 für Informatiker}
\subtitle{Blatt {\blattnr} Hausaufgaben}
\author{
    Florian Abt (6524404), \\
    Lennart Braun (6523742), \\
    Sascha Schulz (6434677)
}
\date{zum 1. Dezember 2015}

\begin{document}
\maketitle

\begin{enumerate}[label=\bfseries \blattnr.\arabic*]


\item 
\begin{enumerate}
\item 
\begin{equation*}
 \begin{split}
    \mathbb{E}_{\sigma^2}[T_n(X_1,\ldots,X_n)]
    &= \mathbb{E}_{\sigma^2}\left[\frac1n \sum_{i=1}^n(X_i-\mu)^2\right] \\
    &= \frac1n \sum_{i=1}^n \mathbb{E}_{\sigma^2} [(X_i-\mu)^2] \\
    &= \frac1n \sum_{i=1}^n Var[X_i] \text{ mit } Var[X_1]=Var[X_i] \\
    &= \frac{n \cdot Var[X_1]}n \\
    &= Var[X_1]
  \end{split}
\end{equation*}

\item 
  Den Schätzer auf Basis des Erwartungswertes, da dies Rechnungsschritte einspart und wir eine
  Information über die Verteilung berücksichtigen, die stichprobenunabhängig ist.
\end{enumerate}

\item 
\begin{enumerate}
\item 

\begin{equation*}
 \begin{split}
   \log \left( L(\vartheta;x_1,\ldots,x_n) \right)
   &= \log \left( \prod_{i=1}^n 2\vartheta x_i e^{-\vartheta x_i^2} \right) \\
   &= \sum_{i=1}^n \left( \log(2) + \log(\vartheta) + \log(x_i) -\vartheta x_i^2 \right) \\
   &= n\log(2) + n\log(\vartheta) + \sum_{i=1}^n x_i^2 - \vartheta \sum_{i=1}^n x_i^2 \\   
   \frac{d}{d\vartheta} \log \left( L(\vartheta;x_1,\ldots,x_n) \right)
   &= \frac{n}\vartheta - \sum_{i=1}^n x_i^2
  \end{split}
\end{equation*}
Daraus folgt als Nullstelle:
\begin{equation*}
\vartheta_0 = \frac{n}{\sum_{i=1}^n x_i^2} 
\end{equation*}
Als Bestätigung, dass es sich bei dem Extrema um ein Maximum handelt, betrachten wir die 2. Ableitung.
\begin{equation*}
 \begin{split} 
   \frac{d}{d\vartheta} \left( \frac{d}{d\vartheta} \log \left( L(\vartheta;x_1,\ldots,x_n) \right) \right)
   = -\frac{n}{\vartheta^2}
   < 0 \text{ für } \vartheta_0 = \frac{n}{\sum_{i=1}^n x_i^2} > 0
  \end{split}
\end{equation*}

\item 
Stichprobe: $x = (11,8,1,6)$.
  
\begin{equation*}
\vartheta_0 = \frac{4}{1^2 + 6^2 + 8^2 + 11^2} = \frac2{111} 
\end{equation*}

\end{enumerate}

\item 
\begin{enumerate}
\item 

\begin{equation*}
 f_\vartheta(x) = \begin{cases}
		    \frac1m &, x \in [1,m] \\
		    0 &, sonst
                  \end{cases}
\end{equation*}

\begin{equation*}
 L(\vartheta;x_1,\ldots,x_n) = \begin{cases}
		    \prod_{i=1}^n\frac1m &, x_1,\ldots,x_n \in [1,m] \\
		    0 &, sonst
                  \end{cases}
\end{equation*}
 


\item 
\end{enumerate}

\item 
\begin{enumerate}
\item 
\item 
\end{enumerate}

\end{enumerate}

\end{document}

