\documentclass[a4paper]{scrartcl}

% font/encoding packages
\usepackage[utf8]{inputenc}
\usepackage[T1]{fontenc}
\usepackage{lmodern}
\usepackage[ngerman]{babel}
\usepackage[ngerman=ngerman-x-latest]{hyphsubst}

\usepackage{amsmath, amssymb, amsfonts, amsthm}
\usepackage{array}
\usepackage{stmaryrd}
\usepackage{marvosym}
\allowdisplaybreaks
\usepackage[output-decimal-marker={,}]{siunitx}
\usepackage[shortlabels]{enumitem}
\usepackage[section]{placeins}
\usepackage{float}
\usepackage{units}
\usepackage{listings}
\usepackage{pgfplots}
\pgfplotsset{compat=1.12}
\usepackage{tikz}
\usetikzlibrary{arrows,automata}

\usepackage{xcolor}
\definecolor{light-gray}{HTML}{cccccc}


\newtheorem*{behaupt}{Behauptung}
\newcommand{\gdw}{\Leftrightarrow}
\newcommand{\dif}{\ \mathrm{d}}
\newcommand{\N}{\mathbb{N}}
\newcommand{\prob}{\mathbb{P}}
\newcommand{\cov}{\operatorname{Cov}}
\newcommand{\e}{\mathbb{E}}
\newcommand{\var}{\operatorname{Var}}
\newcommand{\corr}{\operatorname{Corr}}

\usepackage{fancyhdr}
\pagestyle{fancy}

\lstset{%
    frame=single,
    numbers=left,
    keepspaces,
    language=R,
    title=Listing: \lstname,
}

\def \blattnr {10}

\lhead{Stochastik 2 - Blatt {\blattnr}}
\rhead{Florian Abt, Lennart Braun, Sascha Schulz}
\cfoot{\thepage}


\title{Stochastik 2 für Informatiker}
\subtitle{Blatt {\blattnr} Hausaufgaben}
\author{
    Florian Abt (6524404), \\
    Lennart Braun (6523742), \\
    Sascha Schulz (6434677)
}
\date{zum 5. Januar 2016}

\begin{document}
\maketitle

\begin{enumerate}[label=\bfseries \blattnr.\arabic*]
  \item %.1
    \begin{enumerate}
     \item %.1.a
	~
	% 
	% @see http://www.texample.net/tikz/examples/scatterplot/
	%
	\begin{figure}[H]
	  \centering
	  \begin{tikzpicture}[only marks, y=.5cm]
	    \draw[->] (0,0) -- coordinate (x axis mid) (6,0);
	    \draw[->] (0,0) -- coordinate (y axis mid)(0,11);
	    \foreach \x in {1,2,...,5} {
		\draw [xshift=0cm](\x cm,1pt) -- (\x cm,-3pt)
		    node[anchor=north] {$\x$};
		\draw [light-gray] (\x cm,1pt) -- (\x cm,11);
	    }
	    \foreach \y/\ytext in {1,2,...,10} {
		\draw (1pt, .5*\y cm) -- (-3pt, .5*\y cm) node[anchor=east] {$\ytext$};
		\draw [light-gray] (1pt,.5*\y cm) -- (6,.5*\y cm);
	    }
	    \node[below=0.75cm] at (x axis mid) {Bohnenportion (in 100g)};
	    \node[left=1cm, above=0.75,rotate=90] at (y axis mid) {Flatulenzanzahl};
	    \draw plot[mark=*,xshift=0cm] file {flatulenzen.data};
	  \end{tikzpicture}
	\end{figure}	
     \item %.1.b
     
	\begin{equation*}
	   \begin{split}
	      n&=5 \\
	      \bar{x} &= \frac1n \sum_{i=1}^n x_i = \frac{15}5 = 3 \\
	      \bar{y} &= \frac1n \sum_{i=1}^n y_i = \frac{25}5 = 5 \\
	      s_x^2  &= \frac1{n-1}\left(\sum_{i=1}^n x_i^2 - n\bar{x}^2\right) 
		      = \frac{55-45}{4}
		      = \frac52 \\
	      s_y^2  &= \frac1{n-1}\left(\sum_{i=1}^n y_i^2 - n\bar{y}^2\right) 
		      = \frac{165-125}{4}
		      = 10 \\
	      s_{xy} &= \frac1{n-1}\left(\sum_{i=1}^n x_iy_i - n\bar{x}\bar{y} \right) 
% 		      = \frac{91 - 5\cdot3\cdot5}{5-1} 
		      = \frac{91 - 75}{4}
		      = 4 \\
	      r_{xy} &= \frac{s_{xy}}{s_xs_y} 
		      = \frac{4}{\sqrt{\frac52}\sqrt{10}} 
		      = \frac{4}{\sqrt{\frac{50}2}} 
		      = \frac45
		      = 0.8
	   \end{split}
	\end{equation*}

	Die Bohnenportionen und die Flatulenzen sind positiv korrelliert; der 
	Zusammenhang ist starken Grades.
     \item %.1.c
	Aus vorangegangenem Aufgabenteil wissen wir, dass \\
	$\bar{x} = 3$, $\bar{y} = 5$, $s_x^2 = \frac52$ und $s_{xy} = 4$ ist.
	\begin{equation*}
	   \begin{split}     
	      a &= \frac{s_{xy}}{s_x^2} 
	         = \frac4{\frac52} 
	         = \frac85 
	         = 1.6\\
	      b &= \bar{y} - a\bar{x} 
	         = 5 - 1.6 \cdot 3
	         = 0.2\\
	      y &= ax+b
		 = 1.6x+0.2
	   \end{split}
	\end{equation*}
	% 
	% @see http://www.texample.net/tikz/examples/scatterplot/
	%
	\begin{figure}[H]
	  \centering
	  \begin{tikzpicture}[only marks, y=.5cm, domain=0:4]
	    \draw[->] (0,0) -- coordinate (x axis mid) (6,0);
	    \draw[->] (0,0) -- coordinate (y axis mid)(0,11);
	    \foreach \x in {1,2,...,5} {
		\draw [xshift=0cm](\x cm,1pt) -- (\x cm,-3pt)
		    node[anchor=north] {$\x$};
		\draw [light-gray] (\x cm,1pt) -- (\x cm,11);
	    }
	    \foreach \y/\ytext in {1,2,...,10} {
		\draw (1pt, .5*\y cm) -- (-3pt, .5*\y cm) node[anchor=east] {$\ytext$};
		\draw [light-gray] (1pt,.5*\y cm) -- (6,.5*\y cm);
	    }
	    \node[below=0.75cm] at (x axis mid) {Bohnenportion (in 100g)};
	    \node[left=1cm, above=0.75,rotate=90] at (y axis mid) {Flatulenzanzahl};
	    \draw plot[mark=*,xshift=0cm] file {flatulenzen.data};
	    
	    %\draw (0pt, .5 * .25 cm) -- (5.5cm,.5 * 8.4 cm)
	    \draw (0pt, .5 * .2 cm) -- (5.5cm,.5 * 9 cm)
		  node[right] {$f(x) = 1.6x+0.2$};
	  \end{tikzpicture}
	\end{figure}	
     \item %.1.d
	Gegeben ist $n=81$, $s_x^2 = 50$, $\hat a  = \frac23$, $\hat b=\frac15$, 
	$\hat\sigma^2=40$.
	
	Zu zeigen ist, dass eine positive Korrelation vorliegt, d.h. $a>0$.
	Dies bedeutet im statistischen Test, dass $H_0: a\leq 0$ abzulehnen ist.
	Es gilt daher:
	\begin{equation*}
	   \begin{split}
	      K &= \{ x \in \mathbb{R}: x > t_{n-2;1-\alpha}\} \\
	      \hat \sigma_a^2 &= \frac{\hat\sigma^2}{(n-1)s_x^2}
			      = \frac{40}{80 \cdot 50}
			      = \frac1{100} \\
	      T_n &= \frac{\hat a - a_0}{\hat \sigma _a}
		   = \frac{\frac23 - 0}{\sqrt{\frac1{100}}}
		   = \frac{20}3
		   > 1.9905
		   = t_{79;0.975} \\
	      \Rightarrow T_n &\in K
	   \end{split}
	\end{equation*}
	Somit wird $H_0$ abgelehnt -- es ist eine positive Korrelation belegt.
	
    \end{enumerate}

  \item %.2
    \begin{enumerate}
     \item %.2.a
	\textbf{lm(y\~x)}: Definiert ein \textbf{l}ineares \textbf{M}odell an 
	Hand einer Formel. Die Eingabe y\~x definiert, dass der y Wert in 
	Abhängigkeit zum x Wert steht.
	
	\textbf{plot(x,y)}: Zeichnet die gegebenen Werte der Vektoren 
	gegeneinander auf, wobei die i-te Stelle im x-Vektor mit dem i-ten Wert
	des y-Vektors verknüpft wird.
	
	\textbf{abline(model)}: Zeichnet das lineare Modell, welches zuvor mit 
	lm(y\~x) erstellt wurde in den Plot ein.
     \item %.2.b
     \item %.2.c
     \item %.2.d
    \end{enumerate}
\end{enumerate}

\end{document}

