\documentclass[a4paper]{scrartcl}

% font/encoding packages
\usepackage[utf8]{inputenc}
\usepackage[T1]{fontenc}
\usepackage{lmodern}
\usepackage[ngerman]{babel}
\usepackage[ngerman=ngerman-x-latest]{hyphsubst}

\usepackage{amsmath, amssymb, amsfonts, amsthm}
\usepackage{array}
\usepackage{stmaryrd}
\usepackage{marvosym}
\allowdisplaybreaks
\usepackage[output-decimal-marker={,}]{siunitx}
\usepackage[shortlabels]{enumitem}
\usepackage[section]{placeins}
\usepackage{float}
\usepackage{units}
\usepackage{listings}
\usepackage{pgfplots}
\pgfplotsset{compat=1.12}
\usepackage{tikz}
\usetikzlibrary{arrows,automata}

\newtheorem*{behaupt}{Behauptung}
\newcommand{\gdw}{\Leftrightarrow}
\newcommand{\dif}{\ \mathrm{d}}
\newcommand{\N}{\mathbb{N}}
\newcommand{\prob}{\mathbb{P}}
\newcommand{\cov}{\operatorname{Cov}}
\newcommand{\e}{\mathbb{E}}
\newcommand{\var}{\operatorname{Var}}
\newcommand{\corr}{\operatorname{Corr}}

\usepackage{fancyhdr}
\pagestyle{fancy}

\lstset{%
    frame=single,
    numbers=left,
    keepspaces,
    language=R,
    title=Listing: \lstname,
}

\def \blattnr {5}

\lhead{Stochastik 2 - Blatt {\blattnr}}
\rhead{Florian Abt, Lennart Braun, Sascha Schulz}
\cfoot{\thepage}


\title{Stochastik 2 für Informatiker}
\subtitle{Blatt {\blattnr} Hausaufgaben}
\author{
    Florian Abt (6524404), \\
    Lennart Braun (6523742), \\
    Sascha Schulz (6434677)
}
\date{zum 17. November 2015}

\begin{document}
\maketitle

\begin{enumerate}[label=\bfseries \blattnr.\arabic*]
    \item
        \begin{enumerate}
            \item
	    
	    Ist jedem Zeitschritt kommt 1 Werkstück an und die Maschine fällt mit einer Wahrscheinlichkeit von $p$ aus.
	    
	    Wenn die Warteschlange leer ist, und die Maschine nicht ausfällt, bleibt die Warteschlange leer. Folglich: $p_{0,0}=1-p$.
	    Wenn die Warteschlange voll ist, und die Maschine ausfällt, bleibt die Warteschlange voll. Folglich: $p_{M,M}=p$.

	    Ansonsten gilt: Entweder fällt die Maschine aus, dann verbleibt das angekommene Werkstück in der Warteschleife. 
	    Folglich: $$\forall i \in E\setminus\{M\} \colon p_{i,i+1} = p$$ 
	    Oder aber die Maschine fällt nicht aus, dann wird ein Werkstück der Warteschlange 
	    hinzugefügt und zwei entfernt. Folglich 
	    $$\forall i \in E\setminus\{0\} \colon p_{i,i-1} = 1 - p$$
	    Andere Zustands-Übergänge existieren nicht, daher sind die restlichen Einträge der Matrix 0.

            \item
            
            E ist eine endliche Menge, $(X_n)_{n\in\N_0}$ ist eine $E$-wertige
            DTMC mit Übergangsmatrix $P$. Nach obriger Erläuterung der Zustandsübergänge, wird in einem Zeitschritt stets in den nächsten 
            oder vorherigen Zustand übergegangen; folglich ist jeder Zustand von jedem aus erreichbar und $E$ ist folglich irreduzibel. 
            Folglich ist nach Satz 1.4.1 des Skripts die invariante Verteilung eindeutig.

            \item
            
            Der langfristige Anteil von Zeiteinheiten, in denen der Puffer belegt ist, lässt sich ausdrücken als der langfristige 
            Anteil von Zeiteinheiten, in denen der Puffer nicht leer ist. Folglich:
            \begin{equation*}
                1 - \lim_{n\to\infty} \left( \frac1n \sum_{k=0}^{n-1} \textbf{1}_{\{0\}}(X_k) \right) \stackrel{\text{Satz } 1.5.1}{=} 1 - \pi_0
            \end{equation*}
            Und folglich, basierend auf die gegebene invariante Verteilung aus Aufgabenteil (b):
            \begin{equation*}
	      1 - \pi_0 = \begin{cases}
			    1 - \frac{1-q}{1-q^{M+1}} &,q \neq 1 \\
			    1 - \frac1{M+q} &, q = 1
	                  \end{cases}
            \end{equation*}
            
            Das langfristige Mittel der im Puffer wartenden Werkstücke:
            \begin{equation*}
                \lim_{n \to \infty} \left( \frac{1}{n} \sum_{k=0}^{n-1} X_i \right)
                = \sum_{i=0}^M i \cdot \pi_i
                = \frac{\sum_{i=0}^M i \cdot q^i}{1 + q + q^2 + \dotsb q^M}
            \end{equation*}

        \end{enumerate}
   
    \item
        \begin{enumerate}
            \item
	      Es sei $Y \sim \mathcal{N}(\mu, \sigma^2)$ mit der Dichte
	      \begin{equation*}
		  f_Y(y) = \frac1{\sigma\sqrt{2\pi}} e^{-\frac{(y-\mu)^2}{2\sigma^2}}
	      \end{equation*}
	      
	      Daraus ergibt sich die Verteilungsfunktion $F_Y(y)$ mit:
	    %   \begin{equation*}
		% \begin{split}
		%   F_Y(y) 
		%   &= \int_{-\infty}^y f_Y(t) \dif t \\ 
		%   &= \frac1{\sigma\sqrt{2\pi}} \int_ {-\infty}^y e^{-\frac{(t-\mu)^2}{2\sigma^2}} \dif t\\
		%   &\stackrel{*}{=} \frac1{\sqrt{2\pi}} \int_{-\infty}^{\frac{y-\mu}\sigma} e^{-\frac12z^2} \dif z \\
		%   &= \Phi(y) + \int_y^{(y-\mu)/\sigma} e^{-\frac12z^2} \dif z \text{, mit } \Phi(y) = \frac1{\sqrt{2\pi}} \int_{-\infty}^y e^{\frac12z^2} \dif z \\
		%   & =\Phi\left(\frac{y-\mu}\sigma\right)
		%  \end{split}
	    %   \end{equation*}
	    %   (*) Substitution mit $\varphi(t) = \frac{t-\mu}{\sigma} = z$, \\
	    %   $\varphi'(t) \dif t = \dif z \Leftrightarrow \frac1\sigma \dif t = \dif z \Leftrightarrow dt = \sigma \dif z$.
          \begin{equation*}
              \begin{split}
                  F_Y(y)
                  &= \int_{-\infty}^y \frac{1}{\sigma\sqrt{2\pi}} \cdot
                    e^{-\frac{(x-\mu)^2}{2\sigma^2}} \dif x \\
                  &= \int_{-\infty}^y \frac{1}{\sigma} \cdot \frac{1}{\sqrt{2\pi}}
                    \cdot e^{-\frac{1}{2}\left(\frac{x-\mu}{\sigma}\right)^2}
                    \dif x \\
                  &\stackrel{(\star)}{=} \int_{-\infty}^y \psi'(x) \cdot
                    \frac{1}{\sqrt{2\pi}} \cdot e^{-\frac{1}{2}(\psi(x))^2}
                    \dif x \\
                  &\stackrel{(\star)}{=} \int_{\psi(-\infty)}^{\psi(y)}
                    \frac{1}{\sqrt{2\pi}} \cdot e^{-\frac{1}{2}t^2} \dif t \\
                  &= \int_{-\infty}^{\frac{y-\mu}{\sigma}}
                    \frac{1}{\sqrt{2\pi}} \cdot e^{-\frac{1}{2}t^2} \dif t \\
                  &= \Phi \left( \frac{y-\mu}{\sigma} \right)
              \end{split}
          \end{equation*}
          $(\star)$ Substitution mit $\psi(x) = \frac{x-\mu}{\sigma}$ und
          $\psi'(x) = \frac{1}{\sigma}$.
	      
	      Damit ergibt sich für $\mathbb{P}(a < Y \leq b)$:
	      \begin{equation*}
		\begin{split}
		  \mathbb{P}(a < Y \leq b)
		  &= F_Y(b) - F_Y(a) \\
		  &= \Phi\left( \frac{b - \mu}\sigma\right) - \Phi\left(\frac{a-\mu}\sigma\right) %\\
          % ist glaube ich unnötig
		  % &= \frac1{\sqrt{2\pi}} \int_{-\infty}^{\frac{b - \mu}\sigma} e^{\frac12z^2} \dif z - \left(\frac1{\sqrt{2\pi}} \int_{-\infty}^{\frac{a-\mu}\sigma} e^{\frac12z^2} \dif z \right) \\
		  % &= \frac1{\sqrt{2\pi}} \int_{\frac{a-\mu}\sigma}^{\frac{b - \mu}\sigma} e^{\frac12z^2} \dif z 
		 \end{split}
	      \end{equation*}
	  
	    \item
            \begin{behaupt}
                $\e[X] = 0$ für $X \sim \mathcal{N}(0,1)$.
            \end{behaupt}
            \begin{proof}
                Wir verwenden die Eigenschaft der Standardnormalverteilung,
                dass diese symmetrisch zur Y-Achse ist:
                \begin{equation*}
                    \begin{split}
                        \e[X]
                        &= \int_{-\infty}^\infty x \cdot \varphi(x) \dif x \\
                        &= \int_{-\infty}^\infty x \cdot \frac{1}{\sqrt{2\pi}}
                        \cdot e^{-\frac{x^2}{2}} \dif x \\
                        &= \int_0^\infty x \cdot \frac{1}{\sqrt{2\pi}}
                        \cdot e^{-\frac{x^2}{2}} \dif x +
                        \int_{-\infty}^0 x \cdot \frac{1}{\sqrt{2\pi}}
                        \cdot e^{-\frac{x^2}{2}} \dif x \\
                        &= \int_0^\infty x \cdot \frac{1}{\sqrt{2\pi}}
                        \cdot e^{-\frac{x^2}{2}} \dif x +
                        \int_0^\infty -x \cdot \frac{1}{\sqrt{2\pi}}
                        \cdot e^{-\frac{(-x)^2}{2}} \dif x \\
                        &= \int_0^\infty x \cdot \frac{1}{\sqrt{2\pi}}
                        \cdot e^{-\frac{x^2}{2}} \dif x -
                        \int_0^\infty x \cdot \frac{1}{\sqrt{2\pi}}
                        \cdot e^{-\frac{x^2}{2}} \dif x \\
                        &= 0
                    \end{split}
                \end{equation*}
            \end{proof}
	    % \begin{equation*}
	    %  \begin{split}
		% \mathbb{E}(X)
		% &= \int_{-\infty}^{\infty} \varphi(x) \dif x \\
		% &= \int_{-\infty}^{\infty} \frac1{\sqrt{2\pi}} e^{-\frac12x^2} \dif x \\
		% &= \frac1{\sqrt{2\pi}} \int_{-\infty}^{\infty} e^{-\frac12x^2} \dif x \\
		% &= \frac1{\sqrt{2\pi}} \lim_{a\to -\infty} \lim_{b\to\infty} \left[ - \frac1x \cdot e^{-\frac12 x^2} \right]_a^b \\
		% &= \frac1{\sqrt{2\pi}} \left( \lim_{a\to -\infty} - \frac1  {a \cdot e^{\frac12 a^2}} - \lim_{b\to\infty} - \frac1 {b \cdot e^{\frac12 b^2}} \right) \\
		% &= \frac1{\sqrt{2\pi}} (0 - 0) \\
		% &= 0
	    %  \end{split}
	    % \end{equation*}


        \end{enumerate}
   
    \item
        \begin{enumerate}
            \item
                Sei $Y \sim \text{Bin}(n, p)$. Dann gilt $Y = \sum_{i=1}^n X_i$
                mit $X_i \sim \text{Bin}(1, p)$ für $1 \leq i \leq n$. Die
                $X_i$ sind also iid. mit $\e[X_i] = \mu = p$ und $\var[X_i] =
                \sigma^2 = p(1-p)$.
                Nach dem zentralen Grenzwertsatz ist $\frac{X_1 + \dotsb + X_n
                -n\mu}{\sigma\sqrt{n}}$ approximativ normalverteilt.
                \begin{equation*}
                    \begin{split}
                        \prob(a < Y \leq b)
                        &= \prob \left( a < \sum_{i=1}^n X_i \leq b \right) \\
                        &= \prob \left( a-n\mu < \sum_{i=1}^n X_i -n\mu \leq
                            b-n\mu \right) \\
                        &= \prob \left( \frac{a-n\mu}{\sigma\sqrt{n}} <
                            \frac{\sum_{i=1}^n X_i -n\mu}{\sigma\sqrt{n}} \leq
                            \frac{b-n\mu}{\sigma\sqrt{n}} \right) \\
                            \end{split}
                \end{equation*}  
                Daraus folgt, für $n \to \infty$:       
                \begin{equation*}
                    \begin{split}    
                        &  \lim_{n\to\infty} \prob\left( \frac{a-n\mu}{\sigma\sqrt{n}} <
                            \frac{\sum_{i=1}^n X_i -n\mu}{\sigma\sqrt{n}} \leq
                            \frac{b-n\mu}{\sigma\sqrt{n}} \right) \\
                        &\approx \Phi \left( \frac{b-n\mu}{\sigma\sqrt{n}}
                            \right) - \Phi \left( \frac{a-n\mu}{\sigma\sqrt{n}}
                            \right) \\
                        &= \Phi \left( \frac{b-np}{\sqrt{np(1-p)}} \right) -
                            \Phi \left( \frac{a-np}{\sqrt{np(1-p)}} \right)
                    \end{split}
                \end{equation*}

	      
            \item
                Da es sich bei $M,N$ um ganze Zahlen handelt und
                binomialverteilte Zufallsvariablen nur ganzahlige Werte
                annehmen können, gilt
                \begin{equation*}
                    \prob(M \leq Y \leq N) = \prob(M-1 < Y < N+1)
                    \text{ .}
                \end{equation*}
                Bei der Normalverteilung handelt es sich um eine stetige
                Verteilung. Daher würde bei der Approximation durch diese, ein
                größerer Wert herauskommen, wenn von der rechten Variante
                ausgegangen wird, da das Intervall größer ist, auch wenn es die
                selben Ganzzahlen enthält.  Um auf eine bessere Approximation
                zu kommen wird nun jeweils die Mitte zwischen den beiden
                Möglichkeiten verwendet:
                \begin{equation*}
                    \prob(M-\num{0,5} < Y < N+\num{0,5})
                    \text{ .}
                \end{equation*}

                Da die Normalverteilung Punktwahrscheinlichkeiten wie
                $\prob(Y=y)$ immer den Wert $0$ zuweist, muss für die
                Approximation einer Binomialverteilung ein Intervall um $y$
                angegeben werden:
                \begin{equation*}
                    \prob(Y=y) = \prob(y-\num{0,5} < Y \leq y+\num{0,5})
                \end{equation*}

            \item \hfill \\
                \lstinputlisting{aufgabe-5.3c.R}
                \lstinputlisting{aufgabe-5.3c.out}

        \end{enumerate}
   
\end{enumerate}


\end{document}
