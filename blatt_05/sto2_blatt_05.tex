\documentclass[a4paper]{scrartcl}

% font/encoding packages
\usepackage[utf8]{inputenc}
\usepackage[T1]{fontenc}
\usepackage{lmodern}
\usepackage[ngerman]{babel}
\usepackage[ngerman=ngerman-x-latest]{hyphsubst}

\usepackage{amsmath, amssymb, amsfonts, amsthm}
\usepackage{array}
\usepackage{stmaryrd}
\usepackage{marvosym}
\allowdisplaybreaks
\usepackage[output-decimal-marker={,}]{siunitx}
\usepackage[shortlabels]{enumitem}
\usepackage[section]{placeins}
\usepackage{float}
\usepackage{units}
\usepackage{listings}
\usepackage{pgfplots}
\pgfplotsset{compat=1.12}
\usepackage{tikz}
\usetikzlibrary{arrows,automata}

\newtheorem*{behaupt}{Behauptung}
\newcommand{\gdw}{\Leftrightarrow}
\newcommand{\dif}{\ \mathrm{d}}
\newcommand{\N}{\mathbb{N}}
\newcommand{\prob}{\mathbb{P}}
\newcommand{\cov}{\operatorname{Cov}}
\newcommand{\e}{\mathbb{E}}
\newcommand{\var}{\operatorname{Var}}
\newcommand{\corr}{\operatorname{Corr}}

\usepackage{fancyhdr}
\pagestyle{fancy}

\lstset{%
    frame=single,
    numbers=left,
    keepspaces,
    language=R,
    title=Listing: \lstname,
}

\def \blattnr {5}

\lhead{Stochastik 2 - Blatt {\blattnr}}
\rhead{Florian Abt, Lennart Braun, Sascha Schulz}
\cfoot{\thepage}


\title{Stochastik 2 für Informatiker}
\subtitle{Blatt {\blattnr} Hausaufgaben}
\author{
    Florian Abt (6524404), \\
    Lennart Braun (6523742), \\
    Sascha Schulz (6434677)
}
\date{zum 17. November 2015}

\begin{document}
\maketitle

\begin{enumerate}[label=\bfseries \blattnr.\arabic*]
    \item
        \begin{enumerate}
            \item
	    
	    Ist jedem Zeitschritt kommt 1 Werkstück an und die Maschine fällt mit einer Wahrscheinlichkeit von $p$ aus.
	    
	    Wenn die Warteschlange leer ist, und die Maschine nicht ausfällt, bleibt die Warteschlange leer. Folglich: $p_{0,0}=1-p$.
	    Wenn die Warteschlange voll ist, und die Maschine ausfällt, bleibt die Warteschlange voll. Folglich: $p_{M,M}=p$.

	    Ansonsten gilt: Entweder fällt die Maschine aus, dann verbleibt das angekommene Werkstück in der Warteschleife. 
	    Folglich: $$\forall i \in E\setminus\{M\} \colon p_{i,i+1} = p$$ 
	    Oder aber die Maschine fällt nicht aus, dann wird ein Werkstück der Warteschlange 
	    hinzugefügt und zwei enternt. Folglich 
	    $$\forall i \in E\setminus\{0\} \colon p_{i,i-1} = 1 - p$$
	    Andere Zustands-Übergänge existieren nicht, daher sind die restlichen Einträge der Matrix 0.
	    

	    

	   
            

            \item

            \item

        \end{enumerate}
   
    \item
        \begin{enumerate}
            \item
	      Es sei $Y \sim \mathcal{N}(\mu, \sigma^2)$ mit der Dichte
	      \begin{equation*}
		  f_Y(y) = \frac1{\sigma\sqrt{2\pi}} e^{-\frac{(y-\mu)^2}{2\sigma^2}}
	      \end{equation*}
	      
	      Daraus ergibt sich die Verteilungsfunktion $F_Y(y)$ mit:
	      \begin{equation*}
		\begin{split}
		  F_Y(y) 
		  &= \int_{-\infty}^y f_Y(t) \dif t \\ 
		  &= \frac1{\sigma\sqrt{2\pi}} \int_ {-\infty}^y e^{-\frac{(t-\mu)^2}{2\sigma^2}} \dif t\\
		  &\stackrel{*}{=} \frac1{\sqrt{2\pi}} \int_{-\infty}^{\frac{y-\mu}\sigma} e^{-\frac12z^2} \dif z \\
		  &= \Phi(y) + \int_y^{(y-\mu)/\sigma} e^{-\frac12z^2} \dif z \text{, mit } \Phi(y) = \frac1{\sqrt{2\pi}} \int_{-\infty}^y e^{\frac12z^2} \dif z \\
		  & =\Phi\left(\frac{y-\mu}\sigma\right)
		 \end{split}
	      \end{equation*}
	      (*) Substitution mit $\varphi(t) = \frac{t-\mu}{\sigma} = z$, \\
	      $\varphi'(t) \dif t = \dif z \Leftrightarrow \frac1\sigma \dif t = \dif z \Leftrightarrow dt = \sigma \dif z$.
	      
	      Damit ergibt sich für $\mathbb{P}(a < Y \leq b)$:
	      \begin{equation*}
		\begin{split}
		  \mathbb{P}(a < Y \leq b)
		  &= F_Y(b) - F_Y(a) \\
		  &= \Phi\left( \frac{b - \mu}\sigma\right) - \Phi\left(\frac{a-\mu}\sigma\right) \\
		  &= \frac1{\sqrt{2\pi}} \int_{-\infty}^{\frac{b - \mu}\sigma} e^{\frac12z^2} \dif z - \left(\frac1{\sqrt{2\pi}} \int_{-\infty}^{\frac{a-\mu}\sigma} e^{\frac12z^2} \dif z \right) \\
		  &= \frac1{\sqrt{2\pi}} \int_{\frac{a-\mu}\sigma}^{\frac{b - \mu}\sigma} e^{\frac12z^2} \dif z 
		 \end{split}
	      \end{equation*}
	  
	  \item
	    \begin{equation*}
	     \begin{split}
		\mathbb{E}(X)
		&= \int_{-\infty}^{\infty} \varphi(x) \dif x \\
		&= \int_{-\infty}^{\infty} \frac1{\sqrt{2\pi}} e^{-\frac12x^2} \dif x \\
		&= \frac1{\sqrt{2\pi}} \int_{-\infty}^{\infty} e^{-\frac12x^2} \dif x \\
		&= \frac1{\sqrt{2\pi}} \lim_{a\to -\infty} \lim_{b\to\infty} \left[ - \frac1x \cdot e^{-\frac12 x^2} \right]_a^b \\
		&= \frac1{\sqrt{2\pi}} \left( \lim_{a\to -\infty} - \frac1  {a \cdot e^{\frac12 a^2}} - \lim_{b\to\infty} - \frac1 {b \cdot e^{\frac12 b^2}} \right) \\
		&= \frac1{\sqrt{2\pi}} (0 - 0) \\
		&= 0
	     \end{split}
	    \end{equation*}


        \end{enumerate}
   
    \item
        \begin{enumerate}
            \item
	      Sei $Y = X_1 + \ldots + X_n$.
	      
	      Sei $\chi(X_i)$ die Indintikatorfunktion für die $X_1, \ldots, X_n$, mit
	      \begin{equation*}
	      \chi(X_i) = \begin{cases}
			    1 &, \frac{a}n < X_i \leq \frac{b}n \\
			    0 &, sonst
	                  \end{cases}
	      \end{equation*}

	      
	    \item

            \item

        \end{enumerate}
   
\end{enumerate}


\end{document}
