\documentclass[a4paper]{scrartcl}

% font/encoding packages
\usepackage[utf8]{inputenc}
\usepackage[T1]{fontenc}
\usepackage{lmodern}
\usepackage[ngerman]{babel}
\usepackage[ngerman=ngerman-x-latest]{hyphsubst}

\usepackage{amsmath, amssymb, amsfonts, amsthm}
\usepackage{array}
\usepackage{stmaryrd}
\usepackage{marvosym}
\allowdisplaybreaks
\usepackage[output-decimal-marker={,}]{siunitx}
\usepackage[shortlabels]{enumitem}
\usepackage[section]{placeins}
\usepackage{float}
\usepackage{units}
\usepackage{listings}
\usepackage{pgfplots}
\pgfplotsset{compat=1.12}
\usepackage{tikz}
\usetikzlibrary{arrows,automata}

\newtheorem*{behaupt}{Behauptung}
\newcommand{\gdw}{\Leftrightarrow}
\newcommand{\dif}{\ \mathrm{d}}
\newcommand{\N}{\mathbb{N}}
\newcommand{\prob}{\mathbb{P}}
\newcommand{\cov}{\operatorname{Cov}}
\newcommand{\e}{\mathbb{E}}
\newcommand{\var}{\operatorname{Var}}
\newcommand{\corr}{\operatorname{Corr}}

\usepackage{fancyhdr}
\pagestyle{fancy}

\lstset{%
    frame=single,
    numbers=left,
    keepspaces,
    language=R,
    title=Listing: \lstname,
}

\def \blattnr {4}

\lhead{Stochastik 2 - Blatt {\blattnr}}
\rhead{Florian Abt, Lennart Braun, Sascha Schulz}
\cfoot{\thepage}


\title{Stochastik 2 für Informatiker}
\subtitle{Blatt {\blattnr} Hausaufgaben}
\author{
    Florian Abt (6524404), \\
    Lennart Braun (6523742), \\
    Sascha Schulz (6434677)
}
\date{zum 10. November 2015}

\begin{document}
\maketitle

\begin{enumerate}[label=\bfseries \blattnr.\arabic*]
    \item

    \item
        \begin{enumerate}
            \item
                Es sei $Y \sim \text{U}(0,1)$.
                \begin{behaupt}
                    \begin{equation*}
                        \e[Y] = \frac{1}{2}
                    \end{equation*}
                \end{behaupt}
                \begin{proof}
                    \begin{equation*}
                        \begin{split}
                            \e[Y]
                            &= \int_{-\infty}^\infty y \cdot f_Y(y) \dif y \\
                            &= \int_0^1 y \cdot 1 \dif y \\
                            &= \left[ \frac{1}{2}y^2 \right]_0^1 \\
                            &= \frac{1}{2} - 0 = \frac{1}{2}
                        \end{split}
                    \end{equation*}
                \end{proof}

            \item
                Es sei $Y \sim \text{Exp}(\lambda)$.
                \begin{behaupt}
                    \begin{equation*}
                        \e[Y] = \frac{1}{\lambda}
                    \end{equation*}
                \end{behaupt}
                \begin{proof}
                    Verwendung der Dichte
                    \begin{equation*}
                        \begin{split}
                            \e[Y]
                            &= \int_{-\infty}^\infty y \cdot f_Y(y) \dif y \\
                            &= \int_0^\infty y \cdot
                                \lambda e^{-\lambda y} \dif y \\
                            &= \left[ -y \lambda^2 e^{-\lambda y}
                                \right]_0^\infty
                                + \int_0^\infty e^{-\lambda y} \dif y \\
                            &= \lim_{x \to \infty} \left( -x \cdot \lambda^2
                                e^{-\lambda \cdot x} \right)
                                - \left( 0 \cdot -\lambda^2
                                e^{-\lambda \cdot 0} \right)
                                + \left[ -\frac{1}{\lambda} e^{-\lambda y}
                                \right]_0^\infty \\
                            &= 0 -  0 + \lim_{x \to \infty}
                                \left( -\frac{1}{\lambda} e^{-\lambda \cdot x}
                                \right) - \left( -\frac{1}{\lambda}
                                e^{-\lambda \cdot 0} \right) \\
                            &= 0 - \left(- \frac{1}{\lambda}\right)
                             = \frac{1}{\lambda}
                        \end{split}
                    \end{equation*}
                    Verwendung der Verteilung
                    \begin{equation*}
                        \begin{split}
                            \e[Y]
                            &= \int_0^\infty 1 - F_Y(y) \dif y \\
                            &= \int_0^\infty 1 - (1 - e^{-\lambda y}) \dif y \\
                            &= \int_0^\infty e^{-\lambda y} \dif y \\
                            &= \left[ -\frac{1}{\lambda}e^{-\lambda y}
                                 \right]_0^\infty \\
                            &= \lim_{x \to \infty} \left(
                                 -\frac{1}{\lambda}e^{-\lambda \cdot x} \right) 
                                 -\left( -\frac{1}{\lambda}e^{-\lambda \cdot 0}
                                 \right) \\
                            &= 0 - \left( -\frac{1}{\lambda} \right)
                             = \frac{1}{\lambda}
                        \end{split}
                    \end{equation*}
                \end{proof}

            \item
                Sei $Z$ eine Zufallsvariable mit der Verteilungsfunktion
                $F_Z \colon \mathbb{R} \to [0,1]$:
                \begin{equation*}
                    F_Z(z) =
                    \begin{cases}
                        1 - \alpha e^{-\lambda z} , & z \geq 0 \\
                        0 , & z < 0 \\
                    \end{cases}
                    \text{ für }
                    \alpha \in [0,1]
                \end{equation*}
                \begin{behaupt}
                    \begin{equation*}
                        \e[Z] = \frac{\alpha}{\lambda}
                    \end{equation*}
                \end{behaupt}
                \begin{proof}
                    \begin{equation*}
                        \begin{split}
                            \e[Z]
                            &= \int_0^\infty 1 - F_Z(z) \dif z \\
                            &= \int_0^\infty \alpha e^{-\lambda z} \dif z \\
                            &= \alpha \int_0^\infty e^{-\lambda z} \dif z \\
                            &= \alpha \cdot \frac{1}{\lambda}
                             = \frac{\alpha}{\lambda}
                        \end{split}
                    \end{equation*}
                \end{proof}

        \end{enumerate}

    \item 
        \begin{enumerate}
            \item
                Die Zufallsvariable $Y$ sei mit Parameter $\lambda$ verteilt:
                $Y \sim \text{Exp}(\lambda)$.
                \begin{behaupt}
                    Es gilt $\prob(Y > y) = e^{-\lambda y}$ für $y \geq 0$.
                \end{behaupt}
                \begin{proof}
                    \begin{equation*}
                        \prob(Y > y)
                        = 1 - \prob(Y \leq y)
                        = 1 - F_Y(y)
                        = 1 - (1 - e^{-\lambda y})
                        = e^{-\lambda y}
                    \end{equation*}
                \end{proof}

                \begin{behaupt}
                    Die Exponentialverteilung ist gedächtnislos:
                    Es gilt
                    \begin{equation*}
                        \prob(Y > s + t\ |\ Y > s) = \prob(Y > t)
                        \text{ für }
                        s,t \geq 0
                        \text{ .}
                    \end{equation*}
                \end{behaupt}
                \begin{proof}
                    \begin{equation*}
                        \begin{split}
                            \prob(Y > s + t\ |\ Y > s)
                            &= \frac{\prob(Y > s+t, Y > s)}{\prob(Y > s)} \\
                            &= \frac{\prob(Y > s+t)}{\prob(Y > s)} \\
                            &= \frac{e^{-\lambda (s+t)}}{e^{-\lambda s}} \\
                            &= e^{-\lambda t} \\
                            &= \prob(Y > t)
                        \end{split}
                    \end{equation*}
                \end{proof}

            \item
                Es seien $Y_1, \dotsc, Y_n$ unabhängig mit $Y_i \sim
                \text{Exp}(\lambda_i)$ für $i = 1, \dotsc, n$.
                \begin{behaupt}
                    $\min\{Y_1, \dotsc, Y_n\}$ ist exponentialverteilt mit
                    Parameter $\sum_{i=1}^n \lambda_i$.
                \end{behaupt}
                \begin{proof}
                    \begin{equation*}
                        \begin{split}
                            \prob(\min\{Y_1, \dotsc, Y_n\} > y)
                            &= \prob(Y_1 > y, \dotsc, Y_n > y) \\
                            &= \prob(Y_1 > y) \dotsm \prob(Y_n > y) \\
                            &= e^{-\lambda_1 y} \dotsm e^{-\lambda_n y} \\
                            &= e^{- \left( \sum_{i=1}^n \lambda_i \right) y} \\
                            &\Rightarrow \min\{Y_1, \dotsc, Y_n\} \sim
                            \text{Exp} \left( \sum_{i=1}^n \lambda_i \right)
                        \end{split}
                    \end{equation*}
                \end{proof}

            \item

        \end{enumerate}
   
\end{enumerate}


\end{document}
