\documentclass[a4paper]{scrartcl}

% font/encoding packages
\usepackage[utf8]{inputenc}
\usepackage[T1]{fontenc}
\usepackage{lmodern}
\usepackage[ngerman]{babel}
\usepackage[ngerman=ngerman-x-latest]{hyphsubst}

\usepackage{amsmath, amssymb, amsfonts, amsthm}
\usepackage{array}
\usepackage{stmaryrd}
\usepackage{marvosym}
\allowdisplaybreaks
\usepackage[output-decimal-marker={,}]{siunitx}
\usepackage[shortlabels]{enumitem}
\usepackage[section]{placeins}
\usepackage{float}
\usepackage{units}
\usepackage{listings}
\usepackage{pgfplots}
\pgfplotsset{compat=1.12}
\usepackage{tikz}
\usetikzlibrary{arrows,automata}

\newtheorem*{behaupt}{Behauptung}
\newcommand{\gdw}{\Leftrightarrow}
\newcommand{\N}{\mathbb{N}}
\newcommand{\prob}{\mathbb{P}}
\newcommand{\cov}{\operatorname{Cov}}
\newcommand{\e}{\mathbb{E}}
\newcommand{\var}{\operatorname{Var}}
\newcommand{\corr}{\operatorname{Corr}}

\usepackage{fancyhdr}
\pagestyle{fancy}

\lstset{%
    frame=single,
    numbers=left,
    keepspaces,
    language=R,
    title=Listing: \lstname,
}

\def \blattnr {3}

\lhead{Stochastik 2 - Blatt {\blattnr}}
\rhead{Florian Abt, Lennart Braun, Sascha Schulz}
\cfoot{\thepage}


\title{Stochastik 2 für Informatiker}
\subtitle{Blatt {\blattnr} Hausaufgaben}
\author{
    Florian Abt (6524404), \\
    Lennart Braun (6523742), \\
    Sascha Schulz (6434677)
}
\date{zum 3. November 2015}

\begin{document}
\maketitle

\begin{enumerate}[label=\bfseries \blattnr.\arabic*]
   \item
   \item
   \item 
   
     Eine Invariante Verteilung setzt vorraus, dass $\pi = \pi \cdot P$.
   
     \begin{enumerate}
      \item
	
	\begin{equation*}
	 \pi = \pi \cdot P \\
	 \end{equation*}
	 \begin{equation*}
	  \begin{pmatrix}
	   p1 & p2 & p3
	  \end{pmatrix}
	  =
	  \begin{pmatrix}
	   p1 & p2 & p3
	  \end{pmatrix}
	  \cdot
	  \begin{pmatrix}
	    0 & \frac23 & \frac13 \\
	    \frac13 & 0 & \frac23 \\
	    \frac23 & \frac13 & 0 
	  \end{pmatrix}
	\end{equation*}

      Daraus ergiebt sich das Gleichungssystem:
      
      \begin{equation}
       p1 = \frac23 p2 + \frac13 p3 
      \end{equation}
      \begin{equation}
       p2 = \frac13 p1 + \frac23 p3 
      \end{equation}       
      \begin{equation}
	p3 = \frac23 p1 + \frac13 p2
      \end{equation}       
      
      \begin{equation}
	\begin{split}
       p1 &= \frac23 p2 + \frac13 (p3 = \frac23 p1 + \frac13 p2) \\
       \Rightarrow p1 &= \frac23 p2 + \frac29 p1 + \frac19 p2 \\
       \Rightarrow \frac79 p1 &= \frac79 p2 \\
       \Rightarrow p1 &= p2 \\
       \Rightarrow p3 &= \frac23 p1 + \frac13 p1 \\
       \Rightarrow p3 &= p1
	\end{split}
      \end{equation}
      
      Folglich ist $\pi = (1,1,1)$ die einzige Invariante Verteilung und bis auf eine multiplikative Konstante stets eindeutig.
      
      Da E endlich ist und eine einduetige invariante Verteilung existiert, ist $(X_n)_{n\in\N_0}$ irreduzibel.

      Da $p_{ii}^{(n)} > 0$ für alle $n \in \N$ und alle $i \in E$ ist $(X_n)_{n\in\N_0}$ weder periodisch noch aperiodisch.
	
	
      \item
	Irreduzibilität nicht gegeben, da der Zustand 2 absorbierend 
	und somit von ihm aus gehend kein anderer erreichbar ist. 
	
      \item
	Irreduzibilität nicht gegeben, da die Zustände $\{2,3\}$ absorbierend 
	und somit von ihm aus gehend kein anderer erreichbar ist. 
     \end{enumerate}


\end{enumerate}


\end{document}
