\documentclass[a4paper]{scrartcl}

% font/encoding packages
\usepackage[utf8]{inputenc}
\usepackage[T1]{fontenc}
\usepackage{lmodern}
\usepackage[ngerman]{babel}
\usepackage[ngerman=ngerman-x-latest]{hyphsubst}

\usepackage{amsmath, amssymb, amsfonts, amsthm}
\usepackage{array}
\usepackage{stmaryrd}
\usepackage{marvosym}
\allowdisplaybreaks
\usepackage[output-decimal-marker={,}]{siunitx}
\usepackage[shortlabels]{enumitem}
\usepackage[section]{placeins}
\usepackage{float}
\usepackage{units}
\usepackage{listings}
\usepackage{pgfplots}
\pgfplotsset{compat=1.12}
\usepackage{tikz}
\usetikzlibrary{arrows,automata}

\newtheorem*{behaupt}{Behauptung}
\newcommand{\gdw}{\Leftrightarrow}
\newcommand{\N}{\mathbb{N}}
\newcommand{\prob}{\mathbb{P}}
\newcommand{\cov}{\operatorname{Cov}}
\newcommand{\e}{\mathbb{E}}
\newcommand{\var}{\operatorname{Var}}
\newcommand{\corr}{\operatorname{Corr}}

\usepackage{fancyhdr}
\pagestyle{fancy}

\lstset{%
    frame=single,
    numbers=left,
    keepspaces,
    language=R,
    title=Listing: \lstname,
}

\def \blattnr {2}

\lhead{Stochastik 2 - Blatt {\blattnr}}
\rhead{Florian Abt, Lennart Braun, Sascha Schulz}
\cfoot{\thepage}


\title{Stochastik 2 für Informatiker}
\subtitle{Blatt {\blattnr} Hausaufgaben}
\author{
    Florian Abt (6524404), \\
    Lennart Braun (6523742), \\
    Sascha Schulz (6434677)
}
\date{zum 27. Oktober 2015}

\begin{document}
\maketitle

\begin{enumerate}[label=\bfseries \blattnr.\arabic*]
    \item
        Es sei $X = (X_n)_{n \in \N}$ eine DTMC mit Zustandsraum $E =
        \{1,2,3\}$, Übergangsmatrix
        \begin{equation*}
            P =
            \begin{pmatrix}
                \frac{2}{3} & 0 & \frac{1}{3} \\
                0 & 0 & 1 \\
                \frac{2}{3} & \frac{1}{3} & 0 \\
            \end{pmatrix}
        \end{equation*}
        und Anfangsverteilung $\pi^{(0)} = \alpha$, wobei $\alpha$ ein
        stochastischer Vektor ist.
        \begin{enumerate}[label=\alph*)]
            \item
                \lstinputlisting[%
                ]{aufgabe_2.1.R}

            \item
                Für verschiedene Anfangsverteilungen $\pi^{(0)} = \alpha$
                scheinen die Werte von $\pi^{(n)}$ für $n \to \infty$ gegen den
                gleichen Vektor
                \begin{equation*}
                    \overline{\pi} =
                    \begin{pmatrix}
                        \num{0,6} & \num{0,1} & \num{0,3}
                    \end{pmatrix}
                \end{equation*}
                                                                    
                zu streben. Das langfristige Verhalten dieser DTMC hängt also
                nicht vom Startvektor sondern nur von der Übergangsmatrix $P$
                ab.

        \end{enumerate}

    \item
        Es sei $E$ eine diskrete Menge und $P = (p_{ij})_{i,j \in E}$ und $R =
        (r_{ij})_{i,j \in E}$ seien stochastische Matrizen.
        \begin{enumerate}[label=\alph*)]
            \item
                \begin{behaupt}
                    $P \cdot R$ ist eine stochastische Matrix.
                \end{behaupt}
                \begin{proof}
                    Sei $Q = P \cdot R$. Damit gilt
                    \begin{equation*}
                        q_{ij} = \sum_{k \in E} p_{ik} \cdot r_{kj}
                        \qquad
                        \forall i,j \in E
                    \end{equation*}
                    $Q$ ist eine stochastische Matrix wenn $\forall i,j \in E :
                    q_{ij} \in [0,1]$ und die Einträge der Zeilen sich jeweils
                    zu $1$ aufsummieren:
                    \begin{equation*}
                        \sum_{j \in E} q_{ij} = 1
                        \qquad
                        \forall i \in E
                    \end{equation*}

                    \begin{equation*}
                        \begin{split}
                            \sum_{j \in E} q_{ij}
                            &= \sum_{j \in E} \sum_{k \in E}
                                p_{ik} \cdot r_{kj} \\
                            &= \sum_{k \in E} \sum_{j \in E}
                                p_{ik} \cdot r_{kj} \\
                            &= \sum_{k \in E} p_{ik} \cdot
                                \underbrace{\sum_{j \in E} r_{kj}}_{= 1} \\
                            &= \sum_{k \in E} p_{ik} \\
                            &= 1
                        \end{split}
                    \end{equation*}
                    Da alle Einträge auf $P$ und $R$ nichtnegativ sind, sind
                    es die durch Addition und Multiplikation entstandenen
                    Einträge von $Q$ auch nicht.
                    Da sich die Zeilen zu $1$ aufsummieren, kann jeder Eintrag
                    höchstens den Wert $1$ annehmen.
                    Daher gilt $\forall i,j \in E : q_{ij} \in [0,1]$.
                    $Q = P \cdot R$ ist eine stochastische Matrix.
                \end{proof}

            \item
                \begin{behaupt}
                    Für eine endliche Menge $E$ ist $1$ ein Eigenwert von $P$.
                \end{behaupt}
                \begin{proof}
                    $\lambda$ ist ein Eigenwert von $P$ zum Eigenvektor $x$,
                    wenn gilt
                    \begin{equation*}
                        P \cdot x = \lambda \cdot x
                        \text{ .}
                    \end{equation*}
                    Wir betrachten den Fall $\lambda = 1$. Es muss also einen
                    Spaltenvektor $(x_i)_{i \in E}$ geben, so dass
                    \begin{equation}
                        P \cdot x = x
                        \text{ .}
                        \label{eq:eigenfoo}
                    \end{equation}
                    Wir behaupten, dass der Einsvektor der Länge $|E|$ ein
                    solcher Eigenvektor ist: $x = (1)_{i \in E}$.
                    Sei $x' = P \cdot x$. Dann gilt für alle $i \in E$:
                    \begin{equation*}
                        x'_i = \sum_{j \in E} p_{ij} \cdot x_j
                        = \sum_{j \in E} p_{ij} \cdot 1
                        = \sum_{j \in E} p_{ij}
                        = 1
                        = x_i
                    \end{equation*}
                    Der Einsvektor erfüllt also die Gleichung
                    \eqref{eq:eigenfoo} Damit ist gezeigt, dass $1$ ein
                    Eigenwert von $P$ ist.
                \end{proof}

        \end{enumerate}

    \item
        \begin{enumerate}[label=\alph*)]
            \item
                Es sei $X = (X_n)_{n \in \N_0}$ eine DTMC auf dem Zustandsraum
                $E = \{r,s\}^3$, d.~h. alle Wörter der Länge 3 über dem
                Alphabet $\Sigma = \{r,s\}$. Jeder Zustand repräsentiert die
                drei zuletzt aufgetreteten Farben. Die Übergangsmatrix sei
                gegeben durch $P$. Die Zeilen und Spaltenindizes ergeben sich
                aus den Wörtern, indem diese als Binärzahlen mit $r=0$ und
                $s=1$ interpretiert werden (z.~B.  $rrr = 000_2 = 0$, $srs =
                101_2 = 5$). 
                \begin{equation*}
                    P =
                    \begin{pmatrix}
                        \frac{1}{2} & \frac{1}{2} & 0 & 0 & 0 & 0 & 0 & 0 \\
                        0 & 0 & \frac{1}{2} & \frac{1}{2} & 0 & 0 & 0 & 0 \\
                        0 & 0 & 0 & 0 & \frac{1}{2} & \frac{1}{2} & 0 & 0 \\
                        0 & 0 & 0 & 0 & 0 & 0 & \frac{1}{2} & \frac{1}{2} \\
                        \frac{1}{2} & \frac{1}{2} & 0 & 0 & 0 & 0 & 0 & 0 \\
                        0 & 0 & \frac{1}{2} & \frac{1}{2} & 0 & 0 & 0 & 0 \\
                        0 & 0 & 0 & 0 & \frac{1}{2} & \frac{1}{2} & 0 & 0 \\
                        0 & 0 & 0 & 0 & 0 & 0 & \frac{1}{2} & \frac{1}{2} \\
                    \end{pmatrix}
                \end{equation*}
                Die Markovkette als Zustandsdiagramm:
                \begin{equation*}
                    \begin{tikzpicture}[%
                        ->,
                        >=stealth',
                        shorten >=1pt,
                        auto,
                        node distance=1.5cm,
                        %semithick,
                    ]


                    \node[circle] (0)                 {$rrr$};
                    \node[circle] (1) [right of=0]    {$rrs$};
                    \node[circle] (2) [right of=1]    {$rsr$};
                    \node[circle] (3) [right of=2]    {$rss$};
                    \node[circle] (4) [right of=3]    {$srr$};
                    \node[circle] (5) [right of=4]    {$srs$};
                    \node[circle] (6) [right of=5]    {$ssr$};
                    \node[circle] (7) [right of=6]    {$sss$};

                    \path   (0) edge [loop left]                  node {r} (0)
                            (0) edge [bend right, below]          node {s} (1)
                            (1) edge [bend right]                 node {r} (2)
                            (1) edge [bend right, below]          node {s} (3)
                            (2) edge [bend right]                 node {r} (4)
                            (2) edge [bend right, below]          node {s} (5)
                            (3) edge [bend right, below, pos=0.9] node {r} (6)
                            (3) edge [bend right, below]          node {s} (7)
                            (4) edge [bend right, above]          node {r} (0)
                            (4) edge [bend right, above, pos=0.9] node {s} (1)
                            (5) edge [bend right, above]          node {r} (2)
                            (5) edge [bend right]                 node {s} (3)
                            (6) edge [bend right, above]          node {r} (4)
                            (6) edge [bend right]                 node {s} (5)
                            (7) edge [bend right, above]          node {r} (6)
                            (7) edge [loop right]                 node {s} (7)
                            ;
                    \end{tikzpicture}
                \end{equation*}

                Soll nun ein bestimmter Fall modelliert werden, müssen die
                Wahrscheinlichkeiten in $P$ bzw. die Kanten des Graphen
                angepasst werden. Seien nun $w = rsr$ das gewählte Wort des
                Gegners und $w' = rrs$ das Wort des Teufels. Die folgende
                Matrix $P'$ und der Grap repräsentieren die entsprechende
                DTMC mit den absorbierenden Zuständen $rrs$ und $rsr$.
                \begin{equation*}
                    P' =
                    \begin{pmatrix}
                        \frac{1}{2} & \frac{1}{2} & 0 & 0 & 0 & 0 & 0 & 0 \\
                        0 & 1 & 0 & 0 & 0 & 0 & 0 & 0 \\
                        0 & 0 & 1 & 0 & 0 & 0 & 0 & 0 \\
                        0 & 0 & 0 & 0 & 0 & 0 & \frac{1}{2} & \frac{1}{2} \\
                        \frac{1}{2} & \frac{1}{2} & 0 & 0 & 0 & 0 & 0 & 0 \\
                        0 & 0 & \frac{1}{2} & \frac{1}{2} & 0 & 0 & 0 & 0 \\
                        0 & 0 & 0 & 0 & \frac{1}{2} & \frac{1}{2} & 0 & 0 \\
                        0 & 0 & 0 & 0 & 0 & 0 & \frac{1}{2} & \frac{1}{2} \\
                    \end{pmatrix}
                \end{equation*}
                \begin{equation*}
                    \begin{tikzpicture}[%
                        ->,
                        >=stealth',
                        shorten >=1pt,
                        auto,
                        node distance=1.5cm,
                        %semithick,
                    ]


                    \node[circle] (0)                 {$rrr$};
                    \node[circle] (1) [right of=0]    {$rrs$};
                    \node[circle] (2) [right of=1]    {$rsr$};
                    \node[circle] (3) [right of=2]    {$rss$};
                    \node[circle] (4) [right of=3]    {$srr$};
                    \node[circle] (5) [right of=4]    {$srs$};
                    \node[circle] (6) [right of=5]    {$ssr$};
                    \node[circle] (7) [right of=6]    {$sss$};

                    \path   (0) edge [loop left]                  node {r} (0)
                            (0) edge [bend right, below]          node {s} (1)
                            (1) edge [loop below, below]          node {r} (1)
                            %(1) edge [bend right]                 node {r} (2)
                            %(1) edge [bend right, below]          node {s} (3)
                            (2) edge [loop below, below]          node {r} (2)
                            %(2) edge [bend right]                 node {r} (4)
                            %(2) edge [bend right, below]          node {s} (5)
                            (3) edge [bend right, below, pos=0.9] node {r} (6)
                            (3) edge [bend right, below]          node {s} (7)
                            (4) edge [bend right, above]          node {r} (0)
                            (4) edge [bend right, above, pos=0.9] node {s} (1)
                            (5) edge [bend right, above]          node {r} (2)
                            (5) edge [bend right]                 node {s} (3)
                            (6) edge [bend right, above]          node {r} (4)
                            (6) edge [bend right]                 node {s} (5)
                            (7) edge [bend right, above]          node {r} (6)
                            (7) edge [loop right]                 node {s} (7)
                            ;
                    \end{tikzpicture}
                \end{equation*}

                Für obige Wörter wurde das Skript aus Aufgabe 2.1 entsprechend
                modifiziert. Dabei wurde von einer initialen Gleichverteilung
                ausgegangen:
                \begin{equation*}
                    \pi^{(0)} =
                    \begin{pmatrix}
                        \num{0,125} & 
                        \num{0,125} & 
                        \num{0,125} & 
                        \num{0,125} & 
                        \num{0,125} & 
                        \num{0,125} & 
                        \num{0,125} & 
                        \num{0,125}
                    \end{pmatrix}
                \end{equation*}
                Es ist erkennbar, dass für $n \to \infty$
                \begin{equation*}
                    \pi^{(n)} \to
                    \begin{pmatrix}
                        0 &
                        \frac{2}{3} &
                        \frac{1}{3} &
                        0 &
                        0 &
                        0 &
                        0 &
                        0 &
                    \end{pmatrix}
                    \text{ .}
                \end{equation*}
                Für $n \to \infty$ gilt also $P(X_n = rrs) = \nicefrac{2}{3}$,
                $P(X_n = rsr) = \nicefrac{1}{3}$ und $P(X_n \neq rrs, X_n \neq
                rsr) = 0$.
                Der Teufel gewinnt also mit einer Wahrscheinlichkeit von
                $\nicefrac{2}{3}$.

                \lstinputlisting[%
                ]{aufgabe_2.3.R}

            \item
                Intuitiv würden wir dem Teufel die folgende Strategie
                empfehlen.  Sei $w =a_1a_2a_3 \in \Sigma^3$ mit $\Sigma = \{r,
                s\}$ das vom Gegenspieler gewählte Wort. Nun sollte ein Wort
                $w' = a_0a_1a_2 \in \Sigma^3$ mit $w \neq w'$ gewählt werden.
                Das Suffix von $w'$ soll also dem Prefix von $w$ entsprechen.
                Die Überlegung dahinter ist folgende:
                Sei $\sigma  = \sigma_0\sigma_1\sigma_2 \dotsm \in \Sigma^\omega$ ein unendlich langes Wort über
                $\Sigma$. $\sigma$ representiere die Folge beim Roulette
                beobachteter Zahlen.
                Nun gebe es ein Teilwort $\sigma_i\sigma_{i+1}\sigma_{i+2} = w$,
                welches das erste dieser Art in $\sigma$ ist.
                Ist $i = 0$ so ist $w$ der Beginn von $\sigma$ und der
                Gegenspieler hat gewonnen. Ist $i > 0$, so hat der Teufel die
                Chance zu gewinnen -- wenn $\sigma_{i+1} = a_0$ -- bevor die
                Sequenz des Gegenspielers komplett ist. Denn bevor der Zustand
                $\sigma_1\sigma_2\sigma_3$ erreicht wird, muss sich die
                Markovkette im Zustand $r\sigma_1\sigma_2$ oder
                $s\sigma_1\sigma_2$ befinden.

                Im Beispiel von Teilaufgabe a) hat der Teufel entsprechend
                gewählt und gewinnt mit einer Wahrscheinlichkeit von
                $\nicefrac{2}{3}$.


        \end{enumerate}

\end{enumerate}


\end{document}
