\documentclass[a4paper]{scrartcl}

% font/encoding packages
\usepackage[utf8]{inputenc}
\usepackage[T1]{fontenc}
\usepackage{lmodern}
\usepackage[ngerman]{babel}
\usepackage[ngerman=ngerman-x-latest]{hyphsubst}

\usepackage{amsmath, amssymb, amsfonts, amsthm}
\usepackage{array}
\usepackage{stmaryrd}
\usepackage{marvosym}
\allowdisplaybreaks
\usepackage[output-decimal-marker={,}]{siunitx}
\usepackage[shortlabels]{enumitem}
\usepackage[section]{placeins}
\usepackage{float}
\usepackage{units}
\usepackage{listings}
\usepackage{pgfplots}
\pgfplotsset{compat=1.12}

\newtheorem*{behaupt}{Behauptung}
\newcommand{\gdw}{\Leftrightarrow}
\newcommand{\N}{\mathbb{N}}
\newcommand{\prob}{\mathbb{P}}
\newcommand{\cov}{\operatorname{Cov}}
\newcommand{\e}{\mathbb{E}}
\newcommand{\var}{\operatorname{Var}}
\newcommand{\corr}{\operatorname{Corr}}

\usepackage{fancyhdr}
\pagestyle{fancy}

\def \blattnr {2}

\lhead{Stochastik 2 - Blatt {\blattnr}}
\rhead{Florian Abt, Lennart Braun, Sascha Schulz}
\cfoot{\thepage}


\title{Stochastik 2 für Informatiker}
\subtitle{Blatt {\blattnr} Hausaufgaben}
\author{
    Florian Abt (6524404), \\
    Lennart Braun (6523742), \\
    Sascha Schulz (6434677)
}
\date{zum 27. Oktober 2015}

\begin{document}
\maketitle

\begin{enumerate}[label=\bfseries \blattnr.\arabic*]
    \item

    \item
        Es sei $E$ eine diskrete Menge und $P = (p_{ij})_{i,j \in E}$ und $R =
        (r_{ij})_{i,j \in E}$ seien stochastische Matrizen.
        \begin{enumerate}[label=\alph*)]
            \item
                \begin{behaupt}
                    $P \cdot R$ ist eine stochastische Matrix.
                \end{behaupt}
                \begin{proof}
                    Sei $Q = P \cdot R$. Damit gilt
                    \begin{equation*}
                        q_{ij} = \sum_{k \in E} p_{ik} \cdot r_{kj}
                        \qquad
                        \forall i,j \in E
                    \end{equation*}
                    $Q$ ist eine stochastische Matrix wenn $\forall i,j \in E :
                    q_{ij} \in [0,1]$ und die Einträge der Zeilen sich jeweils
                    zu $1$ aufsummieren:
                    \begin{equation*}
                        \sum_{j \in E} q_{ij} = 1
                        \qquad
                        \forall i \in E
                    \end{equation*}

                    \begin{equation*}
                        \begin{split}
                            \sum_{j \in E} q_{ij}
                            &= \sum_{j \in E} \sum_{k \in E}
                                p_{ik} \cdot r_{kj} \\
                            &= \sum_{k \in E} \sum_{j \in E}
                                p_{ik} \cdot r_{kj} \\
                            &= \sum_{k \in E} p_{ik} \cdot
                                \underbrace{\sum_{j \in E} r_{kj}}_{= 1} \\
                            &= \sum_{k \in E} p_{ik} \\
                            &= 1
                        \end{split}
                    \end{equation*}
                    Da alle Einträge auf $P$ und $R$ nichtnegativ sind, sind
                    es die durch Addition und Multiplikation entstandenen
                    Einträge von $Q$ auch nicht.
                    Da sich die Zeilen zu $1$ aufsummieren, kann jeder Eintrag
                    höchstens den Wert $1$ annehmen.
                    Daher gilt $\forall i,j \in E : q_{ij} \in [0,1]$.
                    $Q = P \cdot R$ ist eine stochastische Matrix.
                \end{proof}

            \item
                \begin{behaupt}
                    Für eine endliche Menge $E$ ist $1$ ein Eigenwert von $P$.
                \end{behaupt}
                \begin{proof}
                    $\lambda$ ist ein Eigenwert von $P$ zum Eigenvektor $x$,
                    wenn gilt
                    \begin{equation*}
                        P \cdot x = \lambda \cdot x
                        \text{ .}
                    \end{equation*}
                    Wir betrachten den Fall $\lambda = 1$. Es muss also einen
                    Spaltenvektor $(x_i)_{i \in E}$ geben, so dass
                    \begin{equation*}
                        P \cdot x = x
                        \text{ .}
                        \label{eq:eigenfoo}
                    \end{equation*}
                    Wir behaupten, dass der Einsvektor der Länge $|E|$ ein
                    solcher Eigenvektor ist: $x = (1)_{i \in E}$.
                    Sei $x' = P \cdot x$. Dann gilt für alle $i \in E$:
                    \begin{equation*}
                        x'_i = \sum_{j \in E} p_{ij} \cdot x_j
                        = \sum_{j \in E} p_{ij} \cdot 1
                        = \sum_{j \in E} p_{ij}
                        = 1
                        = x_i
                    \end{equation*}
                    Der Einsvektor erfüllt also die Gleichung
                    \eqref{eq:eigenfoo} Damit ist gezeigt, dass $1$ ein
                    Eigenwert von $P$ ist.
                \end{proof}

        \end{enumerate}

    \item

\end{enumerate}


\end{document}
