\documentclass[a4paper]{scrartcl}

% font/encoding packages
\usepackage[utf8]{inputenc}
\usepackage[T1]{fontenc}
\usepackage{lmodern}
\usepackage[ngerman]{babel}
\usepackage[ngerman=ngerman-x-latest]{hyphsubst}

\usepackage{amsmath, amssymb, amsfonts, amsthm}
\usepackage{array}
\usepackage{stmaryrd}
\usepackage{marvosym}
\allowdisplaybreaks
\usepackage[output-decimal-marker={,}]{siunitx}
\usepackage[shortlabels]{enumitem}
\usepackage[section]{placeins}
\usepackage{float}
\usepackage{units}
\usepackage{listings}
\usepackage{pgfplots}
\pgfplotsset{compat=1.12}
\usepackage{tikz}
\usetikzlibrary{arrows,automata}

\newtheorem*{behaupt}{Behauptung}
\newcommand{\gdw}{\Leftrightarrow}
\newcommand{\N}{\mathbb{N}}
\newcommand{\prob}{\mathbb{P}}
\newcommand{\cov}{\operatorname{Cov}}
\newcommand{\e}{\mathbb{E}}
\newcommand{\var}{\operatorname{Var}}
\newcommand{\corr}{\operatorname{Corr}}

\usepackage{fancyhdr}
\pagestyle{fancy}

\lstset{%
    frame=single,
    numbers=left,
    keepspaces,
    language=R,
    title=Listing: \lstname,
}

\def \blattnr {2}

\lhead{Stochastik 2 - Blatt {\blattnr}}
\rhead{Florian Abt, Lennart Braun, Sascha Schulz}
\cfoot{\thepage}


\title{Stochastik 2 für Informatiker}
\subtitle{Blatt {\blattnr} Hausaufgaben}
\author{
    Florian Abt (6524404), \\
    Lennart Braun (6523742), \\
    Sascha Schulz (6434677)
}
\date{zum 27. Oktober 2015}

\begin{document}
\maketitle

\begin{enumerate}[label=\bfseries \blattnr.\arabic*]
    \item
        Es sei $X = (X_n)_{n \in \N}$ eine DTMC mit Zustandsraum $E =
        \{1,2,3\}$, Übergangsmatrix
        \begin{equation*}
            P =
            \begin{pmatrix}
                \frac{2}{3} & 0 & \frac{1}{3} \\
                0 & 0 & 1 \\
                \frac{2}{3} & \frac{1}{3} & 0 \\
            \end{pmatrix}
        \end{equation*}
        und Anfangsverteilung $\pi^{(0)} = \alpha$, wobei $\alpha$ ein
        stochastischer Vektor ist.
        \begin{enumerate}[label=\alph*)]
            \item
                \lstinputlisting[%
                ]{aufgabe_2.1.R}

            \item
                Für verschiedene Anfangsverteilungen $\pi^{(0)} = \alpha$
                scheinen die Werte von $\pi^{(n)}$ für $n \to \infty$ gegen den
                gleichen Vektor
                \begin{equation*}
                    \overline{\pi} =
                    \begin{pmatrix}
                        \num{0,6} & \num{0,1} & \num{0,3}
                    \end{pmatrix}
                \end{equation*}
                                                                    
                zu streben. Das langfristige Verhalten hängt also nicht vom
                Startvektor sondern nur von der Übergangsmatrix $P$ ab.

        \end{enumerate}

    \item
        Es sei $E$ eine diskrete Menge und $P = (p_{ij})_{i,j \in E}$ und $R =
        (r_{ij})_{i,j \in E}$ seien stochastische Matrizen.
        \begin{enumerate}[label=\alph*)]
            \item
                \begin{behaupt}
                    $P \cdot R$ ist eine stochastische Matrix.
                \end{behaupt}
                \begin{proof}
                    Sei $Q = P \cdot R$. Damit gilt
                    \begin{equation*}
                        q_{ij} = \sum_{k \in E} p_{ik} \cdot r_{kj}
                        \qquad
                        \forall i,j \in E
                    \end{equation*}
                    $Q$ ist eine stochastische Matrix wenn $\forall i,j \in E :
                    q_{ij} \in [0,1]$ und die Einträge der Zeilen sich jeweils
                    zu $1$ aufsummieren:
                    \begin{equation*}
                        \sum_{j \in E} q_{ij} = 1
                        \qquad
                        \forall i \in E
                    \end{equation*}

                    \begin{equation*}
                        \begin{split}
                            \sum_{j \in E} q_{ij}
                            &= \sum_{j \in E} \sum_{k \in E}
                                p_{ik} \cdot r_{kj} \\
                            &= \sum_{k \in E} \sum_{j \in E}
                                p_{ik} \cdot r_{kj} \\
                            &= \sum_{k \in E} p_{ik} \cdot
                                \underbrace{\sum_{j \in E} r_{kj}}_{= 1} \\
                            &= \sum_{k \in E} p_{ik} \\
                            &= 1
                        \end{split}
                    \end{equation*}
                    Da alle Einträge auf $P$ und $R$ nichtnegativ sind, sind
                    es die durch Addition und Multiplikation entstandenen
                    Einträge von $Q$ auch nicht.
                    Da sich die Zeilen zu $1$ aufsummieren, kann jeder Eintrag
                    höchstens den Wert $1$ annehmen.
                    Daher gilt $\forall i,j \in E : q_{ij} \in [0,1]$.
                    $Q = P \cdot R$ ist eine stochastische Matrix.
                \end{proof}

            \item
                \begin{behaupt}
                    Für eine endliche Menge $E$ ist $1$ ein Eigenwert von $P$.
                \end{behaupt}
                \begin{proof}
                    $\lambda$ ist ein Eigenwert von $P$ zum Eigenvektor $x$,
                    wenn gilt
                    \begin{equation*}
                        P \cdot x = \lambda \cdot x
                        \text{ .}
                    \end{equation*}
                    Wir betrachten den Fall $\lambda = 1$. Es muss also einen
                    Spaltenvektor $(x_i)_{i \in E}$ geben, so dass
                    \begin{equation}
                        P \cdot x = x
                        \text{ .}
                        \label{eq:eigenfoo}
                    \end{equation}
                    Wir behaupten, dass der Einsvektor der Länge $|E|$ ein
                    solcher Eigenvektor ist: $x = (1)_{i \in E}$.
                    Sei $x' = P \cdot x$. Dann gilt für alle $i \in E$:
                    \begin{equation*}
                        x'_i = \sum_{j \in E} p_{ij} \cdot x_j
                        = \sum_{j \in E} p_{ij} \cdot 1
                        = \sum_{j \in E} p_{ij}
                        = 1
                        = x_i
                    \end{equation*}
                    Der Einsvektor erfüllt also die Gleichung
                    \eqref{eq:eigenfoo} Damit ist gezeigt, dass $1$ ein
                    Eigenwert von $P$ ist.
                \end{proof}

        \end{enumerate}

    \pagebreak
    \item
        \begin{enumerate}[label=\alph*)]
            \item
		Sei $\sigma = \sigma_0 \sigma_1 \sigma_2 \dotsm \in \Sigma^\omega$ 
		ein unendlich langes Wort welches die \\$\{rot,schwarz\}$-Ziehung repräsentiert.
		
		Sei $w = \{rot,schwarz\}^3$ ein beliebiges Wort, auf das gewettet werden kann.
		
		Sei $(Y_n)_{n\in\N_0}$ eine DTMC welche angibt, wie viele Übereinstimmungen das gewählte
		Wort im Verlauf von $\sigma$ hat, wobei $Y_i$ die Anzahl der Übereinstimmungen für 
		$\sigma_i$ angibt. Folglich ist $E_Y = \{0,1,2,3\}$, wobei $3\in E$ ein absorbierender 
		Zustand ist und das Vorkommen von $w$ repräsentiert.
		
		Es ergeben sich 4 Grundsätzliche Arten, $w$ zu wählen, wobei für jede Art eine andere
		Übergangsmatrix für die entsprechende $Y_n$ festlegt:
		
		\textbf{1. Fall, Schema ``xxy'': w=(rot,rot,schwarz) $\vee$ w=(schwarz,schwarz,rot)}
		
		\begin{equation*}
                    P_{xxy} =
                    \begin{pmatrix}
                        \frac12 & \frac12 & 0 & 0 \\
                        \frac12 & 0 & \frac12 & 0 \\
                        0 & 0 & \frac12 & \frac12 \\
                        0 & 0 & 0 & 1
                    \end{pmatrix}
                \end{equation*}
                
                \textbf{2. Fall, Schema ``xyx'': w=(rot,schwarz,rot) $\vee$ w=(schwarz,rot,schwarz)}
                
                \begin{equation*}
                    P_{xyx} =
                    \begin{pmatrix}
                        \frac12 & \frac12 & 0 & 0 \\
                        0 & \frac12 & \frac12 & 0 \\
                        \frac12 & 0 & 0 & \frac12 \\
                        0 & 0 & 0 & 1
                    \end{pmatrix}
                \end{equation*}
                
                \textbf{3. Fall, Schema ``xyy'': w=(rot,schwarz,schwarz) $\vee$ w=(schwarz,rot,rot)}
                
                \begin{equation*}
                    P_{xyy} =
                    \begin{pmatrix}
                        \frac12 & \frac12 & 0 & 0 \\
                        0 & \frac12 & \frac12 & 0 \\
                        0 & \frac12 & 0 & \frac12 \\
                        0 & 0 & 0 & 1
                    \end{pmatrix}
                \end{equation*}
                
                \textbf{4. Fall, Schema ``xxx'': w=(rot,rot,rot) $\vee$ w=(schwarz,schwarz,schwarz)}
                
                \begin{equation*}
                    P_{xxx} =
                    \begin{pmatrix}
                        \frac12 & \frac12 & 0 & 0 \\
                        \frac12 & 0 & \frac12 & 0 \\
                        \frac12 & 0 & 0 & \frac12 \\
                        0 & 0 & 0 & 1
                    \end{pmatrix}
                \end{equation*}
                
                Davon abgeleitet, betrachten wir den Verlauf einer Wettrunde zwischen zwei Kontrahenten 
                als DTMC $(X_n)_{n\in\N_0}$. Diese setzt zwei Instanzen von $(Y_n)_{n\in\N_0}$ in Bezug
                und koppelt deren Ereigniswerte; wobei es nur einen Gewinner geben kann. 
                
                Wir stellen dies als Tupel dar, es ergibt sich:
                
                $E_X = \{(a,b)| a,b\in{0,1,2,3} \wedge (a,b)\neq(3,3)\}$, wobei \\
                $E_{absorb} = \{(3,k), (k,3)\}_{k\in\{0,1,2\}}$ absorbierende Zustände sind, \\welche 
                Kenntlich machen, ob $a$ oder $b$ gewonnen hat.
                
                Es ergibt sich die Einschritt-Übergangswahrscheinlichkeit: \\
		$Pr(X_{i+1} = (a',b') | (X_i = (a,b))$ \\
		$ = Pr(Y^A_{i+1} = a', Y^B_{i+1} = b' | Y^A_i = a, Y^B_i=b) $ \\
		$\stackrel{*}{=} Pr(Y^A_{i+1} = a' | Y^A_i = a) \cdot Pr(Y^B_{i+1} = b' | Y^B_i = b)$ \\
		$= p^A_{aa'} \cdot p^B_{bb'}$
                
                (*) da $Y^A$, $Y^B$ stochastisch unabhängig

                
            \item
                Intuitiv würden wir dem Teufel die folgende Strategie empfehlen:
                
                Der Gegenspieler vermutet, dass die Wahl von $w$ beliebig ist, da 
                das eintreten aller $w$ gleichwahrscheinlich ist. Analysiert man die 
                4 Fälle der möglichen Übergangswahrscheinlichkeiten, fällt auf, dass 
                dies nicht korrekt ist, da die Wahrscheinlichkeit von auftretenden Präfixen 
                berücksichtigt werden muss, und nicht das gesamte Wort an sich.
                
                Intuitiv nehmen wir an, dass der Dritte Fall mit dem Schema $xxy$ die beste 
                Gewinnwahrscheinlichkeit mit sich bringt. Daher sollte der Teufel ein Wort dieses
                Falls derart oft wählen, dass der Gegenspieler keinen verdacht schöpft und weiterhin
                an die Gleichverteilung der Wahrscheinlichkeiten glaubt, der Teufel jedoch profitiert.
        \end{enumerate}

\end{enumerate}


\end{document}
