\documentclass[a4paper]{scrartcl}

% font/encoding packages
\usepackage[utf8]{inputenc}
\usepackage[T1]{fontenc}
\usepackage{lmodern}
\usepackage[ngerman]{babel}
\usepackage[ngerman=ngerman-x-latest]{hyphsubst}

\usepackage{amsmath, amssymb, amsfonts, amsthm}
\usepackage{array}
\usepackage{stmaryrd}
\usepackage{marvosym}
\allowdisplaybreaks
\usepackage[output-decimal-marker={,}]{siunitx}
\usepackage[shortlabels]{enumitem}
\usepackage[section]{placeins}
\usepackage{float}
\usepackage{units}
\usepackage{listings}
\usepackage{pgfplots}
\pgfplotsset{compat=1.12}
\usepackage{tikz}
\usetikzlibrary{arrows,automata}

\usepackage{xcolor}
\definecolor{light-gray}{HTML}{cccccc}


\newtheorem*{behaupt}{Behauptung}
\newcommand{\gdw}{\Leftrightarrow}
\newcommand{\dif}{\ \mathrm{d}}
\newcommand{\N}{\mathbb{N}}
\newcommand{\prob}{\mathbb{P}}
\newcommand{\cov}{\operatorname{Cov}}
\newcommand{\e}{\mathbb{E}}
\newcommand{\var}{\operatorname{Var}}
\newcommand{\corr}{\operatorname{Corr}}

\usepackage{fancyhdr}
\pagestyle{fancy}

\lstset{%
    frame=single,
    numbers=left,
    keepspaces,
    language=R,
    title=Listing: \lstname,
}

\def \blattnr {11}

\lhead{Stochastik 2 - Blatt {\blattnr}}
\rhead{Florian Abt, Lennart Braun, Sascha Schulz}
\cfoot{\thepage}


\title{Stochastik 2 für Informatiker}
\subtitle{Blatt {\blattnr} Hausaufgaben}
\author{
    Florian Abt (6524404), \\
    Lennart Braun (6523742), \\
    Sascha Schulz (6434677)
}
\date{zum 12. Januar 2016}

\begin{document}
\maketitle

\begin{enumerate}[label=\bfseries \blattnr.\arabic*]
  \item %.1
    \begin{enumerate}
     \item %.1.a
         Wie in der Vorlesung besprochen gilt:
         \begin{equation*}
             \var\left[T_n^{(1)}\right] = \frac{(b-a)cI - I^2}{n}
         \end{equation*}
         Die Varianz steigt also mit $c$.
         
         % Weiterhin sagt das Skript aus, dass $\frac{T_n^{(1)}}{(b-a)c}$ 
         % binominalverteilt ist, 
         % \\ mit den Parametern $n$ und $p=\frac I{(b-a)c}$.
         % Wird nicht benutzt und ist merkwürdig.
         
         Da $T_n^{(1)}$ ein erwartungstreuer Schätzer für $I$ ist, ist der
         Erwartungswert unabhängig von $c$.
         \begin{equation*}
             \begin{split}
                 \e\left[T_n^{(1)}\right]
                 &= \e \left[ \frac{(b-a)c}{n} \sum_{i=1}^n \mathbf{1}_{[0,g(X_i)]}(Y_i) \right] \\
                 &= (b-a)c \cdot \e \left[ \frac{1}{n} \sum_{i=1}^n \mathbf{1}_{[0,g(X_i)]}(Y_i) \right] \\
                 &= (b-a)c \cdot \prob(Y_1 \leq g(X_1)) \\
                 &= (b-a)c \cdot \frac{I}{(b-a)c} \\
                 &= I
             \end{split}
         \end{equation*}
         
	 Somit ist die Varianz als auch die Konstruktion des Konfidenzintervalls 
	 über die Binominalverteilung von $c$ abhängig, 
	 der Erwartungswert jedoch nicht. 
	 
	 Dies deckt sich mit der Intuition, da 
	 schließlich das zu schätzende Integral nicht davon beeinflusst wird, 
	 wie groß die Fläche ist, aus der Stichproben entnommen werden -- die 
	 Genauigkeit jedoch abnimmt, wenn bei gleichbleibender Stichprobenanzahl
	 die betrachtete Fläche steigt.
     
     \item %.1.b
	 Idealerweise ist $c$ der maximale Wert der Funktion $g(x)$ im Intervall 
	 $[a,b]$, um die Varianz möglichst gering, beziehungsweise das 
	 Konfidenzintervall möglichst schmal zu halten.
     
    \end{enumerate}
  \item %.2
    \begin{enumerate}
     \item %.2.a
         \lstinputlisting{aufgabe-11.2.a.r}
         
         Ergebnisse:
         
         N = 100 $\Rightarrow \frac\pi4 \approx 0.71 \Rightarrow \pi \approx 2.84$, \\
         bei dem Konfidenzintervall [0,6262716 ; 0,7841203].
         
         N = 1000 $\Rightarrow \frac\pi4 \approx 0.76 \Rightarrow \pi \approx 3.04$, \\
         bei dem Konfidenzintervall [0,7367320 ; 0,7821266].
         
         Es ist bereits eine deutliche Annäherung zu sehen. Diese wird mit steigender 
         Stichprobenanzahl um so deutlicher. Da die Verwendung von $runinf(n,min,max)$ 
         allerdings dazu führt, dass das Intervall [0,1] statt [0,1) verwendet wird,
         kann dies zu Störungen führen. Eine bessere Darstellung des Intervalls haben 
         wir bisher nicht gefunden.
     
     \pagebreak
     \item %.2.b
	 \lstinputlisting{aufgabe-11.2.b.r}
         
         Ergebnisse:
         
         N = 100 $\Rightarrow \frac\pi4 \approx 0.73 \Rightarrow \pi \approx 2.92$, \\
         bei dem Konfidenzintervall [0,6473847 ;  0,8020751].
         
         N = 1000 $\Rightarrow \frac\pi4 \approx 0.799 \Rightarrow \pi \approx 3.196$, \\
         bei dem Konfidenzintervall [0,777012 ; 0,819673].
         
         Hier ist bereits eine bessere Annäherung zu sehen und die Konfidenzintervalle 
         sind weniger breit. Dies fügt sich in das Bild aus der Vorlesung, dass für diesen 
         Schätzer bei gleicher Anzahl von Daten bessere Ergebnisse erzielt werden, zu lasten 
         der Rechenzeit.
         
         Die Verwendung von $runinf(n,min,max)$ könnte auch hier Störungen verursacht haben,
         analog zur Aufgabe 11.2.a.
    \end{enumerate}

    \pagebreak
    \item %.3
        Die Verteilungsfunktion $F\colon \mathbb{R} \to [0,1]$ einer mit den
        Parametern $k > 0$ und $x_0 > 0$ Pareto-verteilten Zufallsvariable 
        ist gegeben durch
        \begin{equation*}
            F(x) =
            \begin{cases}
                1 - \left( \frac{x_0}{x} \right)^k & x \geq x_0, \\
                0 & x < x_0. \\
            \end{cases}
        \end{equation*}
        \begin{enumerate}
            \item
                \begin{behaupt}
                    Die Pareto-Verteilung ist stetig und besitzt die Dichte
                    \begin{equation}
                        f(x) =
                        \begin{cases}
                            k \cdot \frac{x_0^k}{x^{k+1}} & x \geq x_0, \\
                            0 & x < x_0. \\
                        \end{cases}
                        \label{eq:dichte}
                    \end{equation}
                \end{behaupt}
                \begin{proof}
                    Auf dem Intervall $(x_0,\infty)$ ist $F$ stetig, da
                    Quotienten, Exponentialfunktionen und Summen mit Konstanten
                    eine stetige Funktion ergeben.
                    Auf $(-\infty,x_0)$ ist $F$ konstant und damit auch stetig.
                    Am Punkt $x = 0$ ist $F$ stetig, da
                    \begin{equation*}
                        \lim_{x \to x_0-} F(x)
                        = 0
                        = 1 - \left( \frac{x_0}{x_0} \right)^k
                        = F(x_0)
                        \text{ .}
                    \end{equation*}
                    Somit ist $F$ auf dem Definitionsbereich stetig.

                    Damit $f$ aus Gleichung \eqref{eq:dichte} die Dichte von
                    $F$ sein kann, muss folgendes gelten:
                    \begin{equation*}
                        \int_0^x f(t) \dif t = F(x)
                        \text{ .}
                    \end{equation*}
                    Fall $x < x_0$:
                    \begin{equation*}
                        \int_0^x f(t) \dif t = 0 = F(x)
                    \end{equation*}
                    Fall $x \geq x_0$:
                    \begin{equation*}
                        \int_0^x f(t) \dif t
                        = \int_{x_0}^x k \cdot \frac{x_0^k}{t^{k+1}} \dif t
                        = \left[ - \left( \frac{x_0}{t} \right)^k \right]_{x_0}^x
                        = 1 - \left( \frac{x_0}{x} \right)^k
                        = F(x)
                    \end{equation*}
                    $f$ ist also die gesuchte Dichte.
                \end{proof}

            \item
                \begin{behaupt}
                    Die Pseudo-Inverse der Verteilungsfunktion ist
                    \begin{equation*}
                        F^{-1}\colon (0,1) \to \mathbb{R}
                        \qquad
                        F^{-1}(y) = \frac{x_0}{\sqrt[k]{1-y}}
                    \end{equation*}
                \end{behaupt}
                \begin{proof}
                    Die Pseudo-Inverse ist definiert als
                    \begin{equation*}
                        F^{-1}(y)
                        = \inf \{ x \in \mathbb{R} : F(x) \geq y \}
                        \text{ .}
                    \end{equation*}
                    Da im zu betrachtenden Intervall (0,1) stets 
                    der Fall $x>x_0$ eintritt, ist $F$ sogar stetig und 
                    streng monoton steigend, sodass die Umkehrfunktion 
                    verwendet werden kann.
                    \begin{equation*}
                        F^{-1}(y) = x
                        \gdw
                        F(x) = y
                        \text{ .}
                    \end{equation*}
                    \begin{equation*}
                        \begin{alignedat}{2}
                            && F(x) &= y \\
                            \gdw\ && 1 - \left( \frac{x_0}{x} \right)^k  &= y \\
                            \gdw\ && 1 - y  &= \left( \frac{x_0}{x} \right)^k \\
                            \Rightarrow\ && \sqrt[k]{1 - y}  &= \frac{x_0}{x} \\
                            \gdw\ && x &= \frac{x_0}{\sqrt[k]{1 - y}} \\
                        \end{alignedat}
                    \end{equation*}
                \end{proof}

            \item
                Es sei $X \sim \mathrm{Pareto}_{k,x_0}$ eine mit den Parametern
                $k$ und $x_0$ Pareto-verteilte Zufallsvariable.
                \begin{behaupt}
                    Der Erwartungswert von $X$ beträgt
                    \begin{equation*}
                        \e[X] = \begin{cases}
				  \frac{k}{k-1} \cdot x_0 &, k>1 \\
				 \infty &, k\leq 1
				\end{cases}
                    \end{equation*}
                \end{behaupt}
                \begin{proof}
                    \begin{equation*}
                        \begin{split}
                            \e[X]
                            &= \int_{-\infty}^\infty t \cdot f(t) \dif t \\
                            &= \int_{x_0}^\infty t \cdot k \cdot \frac{x_0^k}{t^{k+1}} \dif t \\
                            &= \int_{x_0}^\infty k \cdot \frac{x_0^k}{t^k} \dif t \\
                            &= \lim_{a \to \infty} \left[ -\frac{k}{k-1} \cdot \frac{x_0^k}{t^{k-1}} \right]_{x_0}^a \\
                            &= \lim_{a \to \infty} \left( -\frac{k}{k-1} \cdot \frac{x_0^k}{a^{k-1}} \right) - \left( -\frac{k}{k-1} \cdot \frac{x_0^k}{x_0^{k-1}} \right) \\
                            &= \frac{k}{k-1} \cdot \frac{x_0^k}{x_0^{k-1}} -\frac{k}{k-1} \lim_{a \to \infty} \left(\frac{x_0^k}{a^{k-1}} \right) \\    
                            &= \frac{k}{k-1} \left(x_0 -\lim_{a \to \infty} \left(\frac{x_0^k}{a^{k-1}} \right)\right) \\    
			    &= \begin{cases}
				  \frac{k}{k-1} \cdot x_0 &, k>1 \\
				 \infty &, k\leq 1
				\end{cases}
                        \end{split}
                    \end{equation*}
                \end{proof}

        \end{enumerate}
\end{enumerate}

\end{document}

