\documentclass[a4paper]{scrartcl}

% font/encoding packages
\usepackage[utf8]{inputenc}
\usepackage[T1]{fontenc}
\usepackage{lmodern}
\usepackage[ngerman]{babel}
\usepackage[ngerman=ngerman-x-latest]{hyphsubst}

\usepackage{amsmath, amssymb, amsfonts, amsthm}
\usepackage{array}
\usepackage{stmaryrd}
\usepackage{marvosym}
\allowdisplaybreaks
\usepackage[output-decimal-marker={,}]{siunitx}
\usepackage[shortlabels]{enumitem}
\usepackage[section]{placeins}
\usepackage{float}
\usepackage{units}
\usepackage{listings}
\usepackage{pgfplots}
\pgfplotsset{compat=1.12}
\usepackage{tikz}
\usetikzlibrary{arrows,automata}

\usepackage{xcolor}
\definecolor{light-gray}{HTML}{cccccc}


\newtheorem*{behaupt}{Behauptung}
\newcommand{\gdw}{\Leftrightarrow}
\newcommand{\dif}{\ \mathrm{d}}
\newcommand{\N}{\mathbb{N}}
\newcommand{\prob}{\mathbb{P}}
\newcommand{\cov}{\operatorname{Cov}}
\newcommand{\e}{\mathbb{E}}
\newcommand{\var}{\operatorname{Var}}
\newcommand{\corr}{\operatorname{Corr}}

\usepackage{fancyhdr}
\pagestyle{fancy}

\lstset{%
    frame=single,
    numbers=left,
    keepspaces,
    language=R,
    title=Listing: \lstname,
}

\def \blattnr {11}

\lhead{Stochastik 2 - Blatt {\blattnr}}
\rhead{Florian Abt, Lennart Braun, Sascha Schulz}
\cfoot{\thepage}


\title{Stochastik 2 für Informatiker}
\subtitle{Blatt {\blattnr} Hausaufgaben}
\author{
    Florian Abt (6524404), \\
    Lennart Braun (6523742), \\
    Sascha Schulz (6434677)
}
\date{zum 12. Januar 2016}

\begin{document}
\maketitle

\begin{enumerate}[label=\bfseries \blattnr.\arabic*]
    \item %.1
        \begin{enumerate}
            \item

            \item

        \end{enumerate}

    \item %.2
        \begin{enumerate}
            \item

            \item

        \end{enumerate}

    \item %.3
        Die Verteilungsfunktion $F\colon \mathbb{R} \to [0,1]$ einer mit den
        Parametern $k > 0$ und $x_0 > 0$ Paretor-verteilten Zufallsvariable 
        ist gegeben durch
        \begin{equation*}
            F(x) =
            \begin{cases}
                1 - \left( \frac{x_0}{x} \right)^k & x \geq x_0, \\
                0 & x < x_0. \\
            \end{cases}
        \end{equation*}
        \begin{enumerate}
            \item
                \begin{behaupt}
                    Die Pareto-Verteilung ist stetig und besitzt die Dichte
                    \begin{equation}
                        f(x) =
                        \begin{cases}
                            k \cdot \frac{x_0^k}{x^{k+1}} & x \geq x_0, \\
                            0 & x < x_0. \\
                        \end{cases}
                        \label{eq:dichte}
                    \end{equation}
                \end{behaupt}
                \begin{proof}
                    Auf dem Intervall $(x_0,\infty)$ ist $F$ stetig, da
                    Quotienten, Exponentialfunktionen und Summen mit Konstanten
                    eine stetige Funktion ergeben.
                    Auf $(-\infty,x_0)$ ist $F$ konstant und damit auch stetig.
                    Am Punkt $x = 0$ ist $F$ stetig, da
                    \begin{equation*}
                        \lim_{x \to x_0-} F(x)
                        = 0
                        = F(0)
                        = 0
                        = 1 - \left( \frac{x_0}{x_0} \right)^k
                        = \lim_{x \to x_0+} F(x)
                        \text{ .}
                    \end{equation*}
                    Somit ist $F$ auf dem Definitionsbereich stetig.

                    Damit $f$ aus Gleichung \eqref{eq:dichte} die Dichte von
                    $F$ sein kann, muss folgendes gelten:
                    \begin{equation*}
                        \int_0^x f(t) \dif t = F(x)
                        \text{ .}
                    \end{equation*}
                    Fall $x < x_0$:
                    \begin{equation*}
                        \int_0^x f(t) \dif t = 0 = F(x)
                    \end{equation*}
                    Fall $x \geq x_0$:
                    \begin{equation*}
                        \int_0^x f(t) \dif t
                        = \int_{x_0}^x k \cdot \frac{x_0^k}{t^{k+1}} \dif t
                        = \left[ - \left( \frac{x_0}{t} \right)^k \right]_{x_0}^x
                        = 1 - \left( \frac{x_0}{x} \right)^k
                        = F(x)
                    \end{equation*}
                    $f$ ist also die gesuchte Dichte.
                \end{proof}

            \item
                \begin{behaupt}
                    Die Pseudo-Inverse der Verteilungsfunktion ist
                    \begin{equation*}
                        F^{-1}\colon (0,1) \to \mathbb{R}
                        \qquad
                        F^{-1}(y) = \frac{x_0}{\sqrt[k]{1-y}}
                    \end{equation*}
                \end{behaupt}
                \begin{proof}
                    Die Pseudo-Inverse ist definiert als
                    \begin{equation*}
                        F^{-1}(y)
                        = \inf \{ x \in \mathbb{R} : F(x) \geq y \}
                        \text{ .}
                    \end{equation*}
                    Da $F$ eine stetige Funktion ist, gilt sogar
                    \begin{equation*}
                        F^{-1}(y) = x
                        \gdw
                        F(x) = y
                        \text{ .}
                    \end{equation*}
                    \begin{equation*}
                        \begin{alignedat}{2}
                            && F(x) &= y \\
                            \gdw\ && 1 - \left( \frac{x_0}{x} \right)^k  &= y \\
                            \gdw\ && 1 - y  &= \left( \frac{x_0}{x} \right)^k \\
                            \Rightarrow\ && \sqrt[k]{1 - y}  &= \frac{x_0}{x} \\
                            \gdw\ && x &= \frac{x_0}{\sqrt[k]{1 - y}} \\
                        \end{alignedat}
                    \end{equation*}
                \end{proof}

            \item
                Es sei $X \sim \mathrm{Pareto}_{k,x_0}$ eine mit den Parametern
                $k$ und $x_0$ Pareto-verteile Zufallsvariable.
                \begin{behaupt}
                    Der Erwartungswert von $X$ beträgt
                    \begin{equation*}
                        \e[X] = \frac{k}{k-1} \cdot x_0
                    \end{equation*}
                \end{behaupt}
                \begin{proof}
                    TODO: Konvergenz der Integrals
                    \begin{equation*}
                        \begin{split}
                            \e[X]
                            &= \int_{-\infty}^\infty t \cdot f(t) \dif t \\
                            &= \int_{x_0}^\infty t \cdot k \cdot \frac{x_0^k}{t^{k+1}} \dif t \\
                            &= \int_{x_0}^\infty k \cdot \frac{x_0^k}{t^k} \dif t \\
                            &= \left[ -\frac{k}{k-1} \cdot \frac{x_0^k}{t^{k-1}} \right]_{x_0}^\infty \\
                            &= \lim_{t \to \infty} \left( -\frac{k}{k-1} \cdot \frac{x_0^k}{t^{k-1}} \right) - \left( -\frac{k}{k-1} \cdot \frac{x_0^k}{x_0^{k-1}} \right) \\
                            &= \frac{k}{k-1} \cdot x_0
                        \end{split}
                    \end{equation*}
                \end{proof}

        \end{enumerate}

\end{enumerate}

\end{document}

