\documentclass[a4paper]{scrartcl}

% font/encoding packages
\usepackage[utf8]{inputenc}
\usepackage[T1]{fontenc}
\usepackage{lmodern}
\usepackage[ngerman]{babel}
\usepackage[ngerman=ngerman-x-latest]{hyphsubst}

\usepackage{amsmath, amssymb, amsfonts, amsthm}
\usepackage{array}
\usepackage{stmaryrd}
\usepackage{marvosym}
\allowdisplaybreaks
\usepackage[output-decimal-marker={,}]{siunitx}
\usepackage[shortlabels]{enumitem}
\usepackage[section]{placeins}
\usepackage{float}
\usepackage{units}
\usepackage{listings}
\usepackage{pgfplots}
\pgfplotsset{compat=1.12}
\usepackage{tikz}
\usetikzlibrary{arrows,automata}


\newtheorem*{behaupt}{Behauptung}
\newcommand{\gdw}{\Leftrightarrow}
\newcommand{\dif}{\ \mathrm{d}}
\newcommand{\N}{\mathbb{N}}
\newcommand{\prob}{\mathbb{P}}
\newcommand{\cov}{\operatorname{Cov}}
\newcommand{\e}{\mathbb{E}}
\newcommand{\var}{\operatorname{Var}}
\newcommand{\corr}{\operatorname{Corr}}

\usepackage{fancyhdr}
\pagestyle{fancy}

\lstset{%
    frame=single,
    numbers=left,
    keepspaces,
    language=R,
    title=Listing: \lstname,
}

\def \blattnr {6}

\lhead{Stochastik 2 - Blatt {\blattnr}}
\rhead{Florian Abt, Lennart Braun, Sascha Schulz}
\cfoot{\thepage}


\title{Stochastik 2 für Informatiker}
\subtitle{Blatt {\blattnr} Hausaufgaben}
\author{
    Florian Abt (6524404), \\
    Lennart Braun (6523742), \\
    Sascha Schulz (6434677)
}
\date{zum 17. November 2015}

\begin{document}
\maketitle

\begin{enumerate}[label=\bfseries \blattnr.\arabic*]

\item
\begin{enumerate}
 \item 
 
 \begin{equation*}
      \bar{x} 
      = \frac1n \sum_{i=1}^n x_i 
      = \frac1n \sum_{i=1}^k x_i^* \cdot \underbrace{\sum_{i=1}^n \textbf{1}_{\{x_i^*\}}(x_i)}_{h_n(x_i^*)}
      = \sum_{i=1}^k x_i^* \cdot r_n(x_i^*)
 \end{equation*}
 
 \begin{equation*}
  \begin{split}
     s_x^2 
     &= \frac1{n-1} \left( \sum_{i=1}^n x_i^2 - n\bar{x}^2 \right) \\
     &= \frac1{n-1} \left( \sum_{i=1}^k \left( (x_i^*)^2 \cdot \sum_{i=1}^n \textbf{1}_{\{x_i^*\}}(x_i) \right) - n\bar{x}^2  \right) \\
     &= \frac{n}{n-1} \left(\frac1n \sum_{i=1}^k \left( (x_i^*)^2 \cdot \sum_{i=1}^n \textbf{1}_{\{x_i^*\}}(x_i) \right) - \bar{x}^2  \right) \\
     &= \frac{n}{n-1} \left( \sum_{i=1}^k (x_i^*)^2 \cdot r_n(x_i^*) - \bar{x}^2  \right) 
     \end{split}
 \end{equation*}
 
 \item 
 \begin{equation*}
  \begin{split}
      \bar{y} 
      &= \frac1{n+1} \sum_{i=1}^{n+1} x_i  \\
      &= \frac1{n+1} \left( \sum_{i=1}^{n} x_i + x_{n+1} \right) \\
      &= \frac{n}{n+1} \left( \frac1n \sum_{i=1}^{n} x_i + \frac1n x_{n+1} \right) \\
      &= \frac{n}{n+1} \bar{x} + \frac1{n+1} x_{n+1}
  \end{split}
 \end{equation*}
 
\end{enumerate}

\item

\begin{enumerate}
 \item
    Umfang der Stichprobe (4,5,7,7,6,5,7): $n=7$

    Korrekte Sortierung der Stichprobe: (4,5,5,6,7,7,7)
    \begin{align*}
        \text{\~{x}}_{\num{0.2}}
        &= x_{(\lceil n \cdot p \rceil)}
        = x_{(\lceil 7 \cdot 0.2 \rceil)}
        = x_{(\lceil 1.4 \rceil)}
        = x_{(2)} = 5 \\
        \text{\~{x}}_{0.7}
        &= x_{(\lceil n \cdot p \rceil)}
        = x_{(\lceil 7 \cdot 0.7 \rceil)}
        = x_{(\lceil 4.9 \rceil)}
        = x_{(5)} = 7
    \end{align*}

 \item
    $\text{\~{x}}_{0.5}$ zeigt auf das mittlere Element bei ungerader Stichprobengröße und ist somit dessen Median.
    
    Für gerade Stichprobengröße wird jedoch der ``rechte mittlere'' Wert verwendet, was zumindest von unserer Auffassung des 
    Medians im Rahmen dieses Moduls abweicht.
 \item 
 \item 
\end{enumerate}


\item 

\begin{enumerate}
 \item 
 Intuitiv erklärt: Da es sich um eine ungerade Anzahl an Elementen handelt, ist die Position des mittleren Elementes - des Medians - zu wählen.
 Denn: Würde man von diesem Abweichen, so würde dies für $\frac25$ der Parteien eine Distanz sparen, welche jedoch von $\frac35$ der Parteien 
 nun zusätzlich überwunden werden muss. Folglich erhöht sich die Summe und die Position des Medians stellt das Minimum dar.
 
 \item 
 Intuitiv erklärt: Es ist unerheblich, ob als neue Position die bisherige oder eine beliebig andere im [8,17]-Intervall gewählt wird. Die Begründung
 ist analog zu dem vorangegangenen intuitiven Ansatz: Es existieren nun 6 Parteien, also eine gerade Anzahl. 
 Wählen wir eine beliebige Position zwischen den beiden mittleren Elementen, ist ein beliebiges Wegersparnis für die linken $\frac36$ Parteien von den 
 rechten $\frac36$ Parteien zusätzlich zurück zu legen, was in der Summe keinen Unterschied macht. Aus Gründen der Fairness würde vermutlich dennoch der 
 Mittelwert von den zwei mittleren Elementen verwendet werden.
 
\end{enumerate}

\end{enumerate}

\end{document}

