\documentclass[a4paper]{scrartcl}

% font/encoding packages
\usepackage[utf8]{inputenc}
\usepackage[T1]{fontenc}
\usepackage{lmodern}
\usepackage[ngerman]{babel}
\usepackage[ngerman=ngerman-x-latest]{hyphsubst}

\usepackage{amsmath, amssymb, amsfonts, amsthm}
\usepackage{array}
\usepackage{stmaryrd}
\usepackage{marvosym}
\allowdisplaybreaks
\usepackage[output-decimal-marker={,}]{siunitx}
\usepackage[shortlabels]{enumitem}
\usepackage[section]{placeins}
\usepackage{float}
\usepackage{units}
\usepackage{listings}
\usepackage{pgfplots}
\pgfplotsset{compat=1.12}
\usepackage{tikz}
\usetikzlibrary{arrows,automata}


\newtheorem*{behaupt}{Behauptung}
\newcommand{\gdw}{\Leftrightarrow}
\newcommand{\dif}{\ \mathrm{d}}
\newcommand{\N}{\mathbb{N}}
\newcommand{\prob}{\mathbb{P}}
\newcommand{\cov}{\operatorname{Cov}}
\newcommand{\e}{\mathbb{E}}
\newcommand{\var}{\operatorname{Var}}
\newcommand{\corr}{\operatorname{Corr}}

\usepackage{fancyhdr}
\pagestyle{fancy}

\lstset{%
    frame=single,
    numbers=left,
    keepspaces,
    language=R,
    title=Listing: \lstname,
}

\def \blattnr {8}

\lhead{Stochastik 2 - Blatt {\blattnr}}
\rhead{Florian Abt, Lennart Braun, Sascha Schulz}
\cfoot{\thepage}


\title{Stochastik 2 für Informatiker}
\subtitle{Blatt {\blattnr} Hausaufgaben}
\author{
    Florian Abt (6524404), \\
    Lennart Braun (6523742), \\
    Sascha Schulz (6434677)
}
\date{zum 8. Dezember 2015}

\begin{document}
\maketitle

\begin{enumerate}[label=\bfseries \blattnr.\arabic*]

\item 
\begin{enumerate}
 \item 
    $1-\alpha = 0.95 \Rightarrow \alpha = 0.05 \Rightarrow 1-\frac\alpha 2 = 0.975$.
    
    $\Rightarrow z_{1-\frac\alpha 2} = z_{0.975} = 1.960$.
    \begin{equation*}
      \begin{split}
	[U(x_1,\ldots,x_n)&,O(x_1,\ldots,x_n)] \\
	= [\bar{x}-z_{1-\frac\alpha 2} \frac{\sqrt{s_x^2}}{\sqrt{n}} &, \bar{x}+z_{1-\frac\alpha 2} \frac{\sqrt{s_x^2}}{\sqrt{n}}] \\
	= [6.4-1.960 \frac{\sqrt{1.2}}{\sqrt{80}} &, 6.4+1.960 \frac{\sqrt{1.2}}{\sqrt{80}}] \\
	= [6.15995 &, 6.64005]
      \end{split}
    \end{equation*}
 \item 
    Nun sei $\sigma^2=0.6$.    
    \begin{equation*}
      \begin{split}
	O(x_1,\ldots, x_n) - U(x_1,\ldots,x_n) 
	= l 
	&= O(x_1,\ldots, x_m) - U(x_1,\ldots,x_m) \\
	&= \left(\bar{x} + z_{1-\frac\alpha 2}\frac\sigma{\sqrt{m}}\right) - \left(\bar{x} - z_{1-\frac\alpha 2}\frac\sigma{\sqrt{m}}\right) \\
	&= 2z_{1-\frac\alpha 2}\frac\sigma{\sqrt{m}} \\
	\Rightarrow \sqrt{m} &= 2z_{1-\frac\alpha 2}\frac\sigma{l} \\
	\Rightarrow m &= 4z^2_{1-\frac\alpha 2}\frac{\sigma^2}{l^2} \\
	&=4 \cdot (1.960)^2 \cdot \frac{0.6}{(6.64005 - 6.15995)^2} \\
	&=40
      \end{split}
    \end{equation*}
    Der Ingenieur hätte nur eine Stichprobe vom Umfang $m=40$ testen müssen.
 \item 
  Nun sei l=0.25, gesucht $1-\alpha$.
  \begin{equation*}
      \begin{split}
	l 
	&=O(x_1,\ldots, x_n) - U(x_1,\ldots,x_n)  \\
	&= \left(\bar{x} + z_{1-\frac\alpha 2}\frac\sigma{\sqrt{n}}\right) - \left(\bar{x} - z_{1-\frac\alpha 2}\frac\sigma{\sqrt{n}}\right) \\
	&= 2z_{1-\frac\alpha 2}\frac\sigma{\sqrt{n}} \\
	\Rightarrow z_{1-\frac\alpha 2} &= \frac {l\cdot \sqrt{n}}{2\sigma} \\
	\Rightarrow 1-\frac\alpha 2 &= \Phi\left( \frac {l\cdot \sqrt{n}}{2\sigma} \right) \\
	\Rightarrow 1-\alpha &= 2\cdot \Phi\left( \frac {l\cdot \sqrt{n}}{2\sigma} \right) -1 \\
	&= 2\cdot \Phi\left( \frac{0.25 \cdot \sqrt{80}}{2 \sqrt{0.6}} \right) -1 \\
	&= 2\cdot \Phi(1.443376) -1 \\
	&\approx 2 \cdot \Phi(1.44) -1 \\
	&= 2 \cdot 0.9251 - 1 \\
	&= 0.8502
      \end{split}
    \end{equation*}
    Das Konfidenzintevall des Kollegen genügt lediglich dem Niveau von ca. 0.85.
\end{enumerate}

\item 

\begin{enumerate}
 \item 
 \item 
  Konfidenzniveau 0.95:
  
  $1-\alpha = 0.95 \Rightarrow \alpha=0.05 \Rightarrow \frac\alpha 2 = 0.025 \Rightarrow 1-\frac\alpha 2 = 0.975$.
  
  $\chi_{n-1;\alpha/2} = \chi_{8;0.05} = 2.18$, $\chi_{n-1;1-\alpha/2} = \chi_{8;0.975} = 17.53$.
  \begin{equation*}
    \begin{split}
      \left[ \frac{(n-1)s_x^2}{\chi^2_{n-1;1-\alpha /2}} \right.,\left. \frac{(n-1)s_x^2}{\chi^2_{n-1;\alpha /2}} \right] 
      \Rightarrow \left[ \frac{(9-1) \cdot 6.25}{17.53} \right.,\left. \frac{(9-1) \cdot 6.25}{2.18} \right] 
      = [2.852253 , 22.93578]
    \end{split}
  \end{equation*}
  
  Konfidenzniveau 0.99:
  
  $1-\alpha = 0.99 \Rightarrow \alpha=0.01 \Rightarrow \frac\alpha 2 = 0.005 \Rightarrow 1-\frac\alpha 2 = 0.995$.
  
  $\chi_{n-1;\alpha/2} = \chi_{8;0.01} = 1.34$, $\chi_{n-1;1-\alpha/2} = \chi_{8;0.995} = 21.96$.
  \begin{equation*}
    \begin{split}
      \left[ \frac{(n-1)s_x^2}{\chi^2_{n-1;1-\alpha /2}} \right.,\left. \frac{(n-1)s_x^2}{\chi^2_{n-1;\alpha /2}} \right] 
      \Rightarrow \left[ \frac{(9-1) \cdot 6.25}{21.96} \right.,\left. \frac{(9-1) \cdot 6.25}{1.34} \right] 
      = [2.276867 , 37.31343]
    \end{split}
  \end{equation*}
\end{enumerate}

\item 
\begin{enumerate}
 \item 
 \item 
\end{enumerate}

\end{enumerate}

\end{document}

