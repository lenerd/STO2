\documentclass[a4paper]{scrartcl}

% font/encoding packages
\usepackage[utf8]{inputenc}
\usepackage[T1]{fontenc}
\usepackage{lmodern}
\usepackage[ngerman]{babel}
\usepackage[ngerman=ngerman-x-latest]{hyphsubst}

\usepackage{amsmath, amssymb, amsfonts, amsthm}
\usepackage{array}
\usepackage{stmaryrd}
\usepackage{marvosym}
\allowdisplaybreaks
\usepackage[output-decimal-marker={,}]{siunitx}
\usepackage[shortlabels]{enumitem}
\usepackage[section]{placeins}
\usepackage{float}
\usepackage{units}
\usepackage{listings}
\usepackage{pgfplots}
\pgfplotsset{compat=1.12}
\usepackage{tikz}
\usetikzlibrary{arrows,automata}


\newtheorem*{behaupt}{Behauptung}
\newcommand{\gdw}{\Leftrightarrow}
\newcommand{\dif}{\ \mathrm{d}}
\newcommand{\N}{\mathbb{N}}
\newcommand{\prob}{\mathbb{P}}
\newcommand{\cov}{\operatorname{Cov}}
\newcommand{\e}{\mathbb{E}}
\newcommand{\var}{\operatorname{Var}}
\newcommand{\corr}{\operatorname{Corr}}

\usepackage{fancyhdr}
\pagestyle{fancy}

\lstset{%
    frame=single,
    numbers=left,
    keepspaces,
    language=R,
    title=Listing: \lstname,
}

\def \blattnr {8}

\lhead{Stochastik 2 - Blatt {\blattnr}}
\rhead{Florian Abt, Lennart Braun, Sascha Schulz}
\cfoot{\thepage}


\title{Stochastik 2 für Informatiker}
\subtitle{Blatt {\blattnr} Hausaufgaben}
\author{
    Florian Abt (6524404), \\
    Lennart Braun (6523742), \\
    Sascha Schulz (6434677)
}
\date{zum 8. Dezember 2015}

\begin{document}
\maketitle

\begin{enumerate}[label=\bfseries \blattnr.\arabic*]

    \item 
        Es sei x eine Stichprobe vom Umfang $n = 80$ mit Mittelwert
        $\overline{x} = \num{6,4}$ und Stichprobenvarianz $s_x^2 = \num{1,2}$.
        \begin{enumerate}
            \item 
                \begin{behaupt}
                    $[\num{6,1562}, \num{6,6438}]$
                    ist ein Konfidenzintervall für den Erwartungswert $\mu$ zum
                    Niveau \num{0,95}.
                \end{behaupt}
                \begin{proof}
                    \begin{gather*}
                        1 - \alpha = \num{0.95}
                        \gdw
                        \alpha = \num{0.05}
                        \gdw
                        1 - \frac{\alpha}{2} = \num{0.975}
                        \\
                        \Rightarrow t_{n-1; 1-\frac{\alpha}{2}}
                        =
                        t_{79; \num{0.975}} = \num{1.9905}
                    \end{gather*}
                    \begin{equation*}
                        \begin{split}
                            \left[ U(x_1, \dotsc, x_n),\ 
                            O(x_1, \dotsc, x_n) \right]
                            &= \left[ \overline{x} - t_{n-1; 1-\frac{\alpha}{2}}
                            \frac{\sqrt{s_x^2}}{\sqrt{n}},\ 
                            \overline{x} + t_{n-1; 1 - \frac{\alpha}{2}}
                            \frac{\sqrt{s_x^2}}{\sqrt{n}} \right] \\
                            &= \left[ \num{6.4} - \num{1.960}
                            \frac{\sqrt{\num{1.2}}}{\sqrt{80}},\ 
                            \num{6.4} + \num{1.960}
                            \frac{\sqrt{\num{1.2}}}{\sqrt{80}} \right] \\
                            &= \left[ \num{6.15995},\ \num{6.64005} \right]
                        \end{split}
                    \end{equation*}
                \end{proof}
    
            \item 
                Nun sei $\sigma^2 = \num{0.6}$.
                \begin{behaupt}
                    Der Ingenieur hätte nur eine Stichprobe vom Umfang $m = 49$
                    testen müssen.
                \end{behaupt}
                \begin{proof}
                    Die Breite des Konfidenzintervalls ist
                    \begin{equation*}
                        b = 2 \cdot t_{n-1; 1-\frac{\alpha}{2}}\frac{s_x}{\sqrt{n}}
                        = \num{0.4876}
                        \text{ .}
                    \end{equation*}
                    Da die Stichprobenlänge eine natürliche Zahl sein muss, ist
                    es unter Umständen sehr schwierig, dieselbe
                    Konfidenzintervall-Länge zu erreichen.
                    Daher gehen wir davon aus, dass das neue Konfidenzintervall
                    das breiteste sein soll, welches mindestens so gut (schmal)
                    sein muss, wie das alte.
                    \begin{equation*}
                        \begin{alignedat}{2}
                            && b 
                            &\geq 2 \cdot z_{1-\frac{\alpha}{2}}
                            \frac{\sigma}{\sqrt{m}} \\
                            \gdw\ && \sqrt{m}
                            &\geq 2 \cdot z_{1-\frac{\alpha}{2}}
                            \frac{\sigma}{b} \\
                            \Rightarrow\ && m
                            &\geq 4 \cdot z_{1-\frac{\alpha}{2}}^2
                            \frac{\sigma^2}{b^2} \\
                            \Rightarrow\ && m
                            &= \left\lceil 4 \cdot z_{1-\frac{\alpha}{2}}^2
                            \frac{\sigma^2}{b^2} \right\rceil \\
                            \gdw\ && m
                            &= \left\lceil \num{38,7789} \right\rceil \\
                            \gdw\ && m &= 39 \\
                        \end{alignedat}
                    \end{equation*}
                \end{proof}

            \item 
                \begin{behaupt}
                    Das Konfidenzintervall der Breite $b' = \num{0,25}$ bei
                    einem Stichprobenumfang von $n' = 80$ genügte einem
                    Konfidenzniveau von $1 - \alpha' = \num{0,8502}$.
                \end{behaupt}
                \begin{proof}
                    \begin{equation*}
                        \begin{alignedat}{2}
                            && b' &= 2 \cdot z_{1-\alpha'}
                            \frac{\sigma}{\sqrt{n'}} \\
                            \gdw\ && z_{1-\alpha'}
                            &= \frac{b' \sqrt{n'}}{2 \sigma} \\
                            \gdw\ && 1-\alpha'
                            &= \Phi \left( \frac{b' \sqrt{n'}}{2 \sigma}
                            \right) \\
                            &&&= \Phi \left( \frac{\num{0,25} \sqrt{80}}
                            {2 \sqrt{\num{0,6}}} \right) \\
                            &&&\approx \Phi \left( \num{1,4434} \right) \\
                            &&&\approx \num{0,8502}
                        \end{alignedat}
                    \end{equation*}
                \end{proof}
        \end{enumerate}

    \item 
        \begin{enumerate}
            \item 
                \begin{behaupt}
                    \begin{equation*}
                        \left[
                            \frac{(n-1)S_X^2}
                            {\chi^2_{n-1; 1-\nicefrac{\alpha}{2}}}
                            ,
                            \frac{(n-1)S_X^2}
                            {\chi^2_{n-1; \nicefrac{\alpha}{2}}}
                        \right]
                    \end{equation*}
                    ist ein Konfidenzinterfall für $\sigma^2$ zum
                    Konfidenzniveau $1 - \alpha$.
                \end{behaupt}
                \begin{proof}
                    \begin{equation*}
                        \begin{split}
                            \prob
                            \left(
                                \frac{(n-1)S_X^2}
                                {\chi^2_{n-1; 1-\nicefrac{\alpha}{2}}}
                                \leq
                                \sigma^2
                                \leq
                                \frac{(n-1)S_X^2}
                                {\chi^2_{n-1; \nicefrac{\alpha}{2}}}
                            \right)
                            &=
                            \prob
                            \left(
                                \frac{\chi^2_{n-1; \nicefrac{\alpha}{2}}}
                                {(n-1)S_X^2}
                                \leq
                                \frac{1}{\sigma^2}
                                \leq
                                \frac{\chi^2_{n-1; 1-\nicefrac{\alpha}{2}}}
                                {(n-1)S_X^2}
                            \right) \\
                            &=
                            \prob
                            \left(
                                \chi^2_{n-1; \nicefrac{\alpha}{2}}
                                \leq
                                \frac{(n-1)S_X^2}{\sigma^2}
                                \leq
                                \chi^2_{n-1; 1-\nicefrac{\alpha}{2}}
                            \right) \\
                            &=
                            \prob
                            \left(
                                \frac{(n-1)S_X^2}{\sigma^2}
                                \leq
                                \chi^2_{n-1; 1-\nicefrac{\alpha}{2}}
                            \right) \\
                            &-
                            \prob
                            \left(
                                \frac{(n-1)S_X^2}{\sigma^2}
                                \leq
                                \chi^2_{n-1; \nicefrac{\alpha}{2}}
                            \right) \\
                            &= 1- \frac{\alpha}{2} - \frac{\alpha}{2}
                            = 1- \alpha
                        \end{split}
                    \end{equation*}
                \end{proof}

            \item 
                Es seien $n = 9$ und $s_x^2 = \num{6,25}$.

                Konfidenzniveau $1 - \alpha = \num{0.95}$:
  
                \begin{equation*}
                    1 - \alpha = \num{0.95}
                    \gdw
                    \alpha = \num{0.05}
                    \gdw
                    \frac{\alpha}{2} = \num{0.025}
                    \gdw
                    1 - \frac{\alpha}{2} = \num{0.975}
                \end{equation*}
                \begin{equation*}
                    \chi_{n-1; \nicefrac{\alpha}{2}}^2
                    = \chi_{8; \num{0.025}}^2
                    = \num{2.18}
                    \qquad
                    \chi_{n-1; 1 - \nicefrac{\alpha}{2}}^2
                    = \chi_{8; \num{0.975}}^2
                    = \num{17.53}
                \end{equation*}
                \begin{equation*}
                    \left[
                    \frac{(n-1)s_x^2}{\chi^2_{n-1;1-\frac{\alpha}{2}}}
                    ,\ 
                    \frac{(n-1)s_x^2}{\chi^2_{n-1;\frac{\alpha}{2}}}
                    \right] 
                    =
                    \left[
                    \frac{(9-1) \cdot \num{6.25}}{\num{17.53}}
                    ,\ 
                    \frac{(9-1) \cdot \num{6.25}}{\num{2.18}}
                    \right] 
                    = [\num{2.8523},\ \num{22.9358}]
                \end{equation*}
  
                Konfidenzniveau $1 - \alpha' = \num{0.99}$:
  
                \begin{equation*}
                    1 - \alpha' = \num{0.99}
                    \gdw
                    \alpha' = \num{0.01}
                    \gdw
                    \frac{\alpha'}{2} = \num{0.005}
                    \gdw
                    1 - \frac{\alpha'}{2} = \num{0.995}
                \end{equation*}
                \begin{equation*}
                    \chi_{n-1; \nicefrac{\alpha'}{2}}^2
                    = \chi_{8; \num{0.005}}^2
                    = \num{1.34}
                    \qquad
                    \chi_{n-1; 1 - \nicefrac{\alpha'}{2}}^2
                    = \chi_{8; \num{0.995}}^2
                    = \num{21.96}
                \end{equation*}
                \begin{equation*}
                    \left[
                    \frac{(n-1)s_x^2}{\chi^2_{n-1;1-\frac{\alpha'}{2}}}
                    ,\ 
                    \frac{(n-1)s_x^2}{\chi^2_{n-1;\frac{\alpha'}{2}}}
                    \right] 
                    =
                    \left[
                    \frac{(9-1) \cdot \num{6.25}}{\num{21,96}}
                    ,\ 
                    \frac{(9-1) \cdot \num{6.25}}{\num{1,34}}
                    \right] 
                    = [\num{2,2769},\ \num{37,3134}]
                \end{equation*}
        \end{enumerate}

    \pagebreak
    \item
        \begin{enumerate}
            \item \hfill \\ 
                \lstinputlisting{aufgabe-8.3.a.r}

            \item 
                Der ermittelte relative Anteil war \num{0.9009} und passt
                soweit zum angegebenen Konfidenzniveau von \num{0.9}.

        \end{enumerate}

\end{enumerate}

\end{document}

